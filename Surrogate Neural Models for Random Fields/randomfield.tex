\documentclass{article}

% if you need to pass options to natbib, use, e.g.:
%     \PassOptionsToPackage{numbers, compress}{natbib}
% before loading neurips_2024


% ready for submission
\usepackage[final]{neurips_2024}


% to compile a preprint version, e.g., for submission to arXiv, add add the
% [preprint] option:
%     \usepackage[preprint]{neurips_2024}


% to compile a camera-ready version, add the [final] option, e.g.:
%     \usepackage[final]{neurips_2024}


% to avoid loading the natbib package, add option nonatbib:
%    \usepackage[nonatbib]{neurips_2024}


\usepackage[utf8]{inputenc} % allow utf-8 input
\usepackage[T1]{fontenc}    % use 8-bit T1 fonts
\usepackage{hyperref}       % hyperlinks
\usepackage{url}            % simple URL typesetting
\usepackage{booktabs}       % professional-quality tables
\usepackage{amsfonts}       % blackboard math symbols
\usepackage{nicefrac}       % compact symbols for 1/2, etc.
\usepackage{microtype}      % microtypography
\usepackage{xcolor}         % colors
\usepackage{amsmath}
\usepackage{amsthm}
\usepackage{mathrsfs}
\usepackage{graphicx}
\usepackage{caption}
\usepackage{float}


\newtheorem{theorem}{Theorem}
\newtheorem{definition}{Definition}
\newtheorem{remark}{Remark}
\newtheorem{example}{Example}

\title{Modeling Neural Random Field through Operator-based Diffusion}


% The \author macro works with any number of authors. There are two commands
% used to separate the names and addresses of multiple authors: \And and \AND.
%
% Using \And between authors leaves it to LaTeX to determine where to break the
% lines. Using \AND forces a line break at that point. So, if LaTeX puts 3 of 4
% authors names on the first line, and the last on the second line, try using
% \AND instead of \And before the third author name.


\author{%
    Guanyu Chen \\
    College of Integrated Circuits\\
    Zhejiang Uinversity\\
    Hangzhou, Zhejiang\\
    \texttt{guanyu@zju.edu.cn} 
}


\begin{document}


\maketitle


\begin{abstract}

\end{abstract}


\section{Introduction}
Some references: \cite{whittle1954stationary, carrizo2022general, lindgren2020diffusion, sigrist2015stochastic, bolin2020rational, Porcu2023}
\begin{definition}
    Random field $\mathcal{M}(x, \omega)$, where $x\in D$ and $\omega \in \Omega$, is defined as:
\begin{equation}
    \begin{aligned}
        & \mathcal{M}(x, \cdot) \text{ is a random variable defined on the probability space } (\Omega, \mathcal{F}, P),\\
        & \mathcal{M}(\cdot, \omega) \text{ is a deterministic function of } x.
    \end{aligned}
\end{equation}
\end{definition}
Classical methods to simulate random field are based on polynomial chaos expansion \ref{PCE} and Karhunen-Loeve expansion \ref{KLE}, See \cite{Lord_Powell_Shardlow_2014}.
Random field can be regarded as a Hilbert space($L^2(\Omega, H)$)-valued random variable.
\begin{definition}[$L^p(\Omega, H)$ space]
  Let $(\Omega, \mathcal{F}, \mathbb{P})$ be a probability space and $H$ is a Hilbert space with norm $\|\cdot\|$. Then $\mathcal{L}^p(\Omega, H)$ with $1\leq p<\infty$ is the space
  of H-valued $\mathcal{F}$-measurable random vaiables $X:\Omega\rightarrow H$ with $\mathbf{E}[\|X\|^p]<\infty$ and a Banach space with norm:
  \begin{equation}
      \|X\|_{\mathcal{L}^p(\Omega, H)}:=\left(\int_\Omega \|X(\omega)\|^pdP(\omega)\right)^{\frac{1}{p}}=\mathbf{E}[\|X\|^p]^{\frac{1}{p}}
  \end{equation}
\end{definition}
Then we can define the inner product: 
\begin{equation}
  \langle X, Y\rangle_{\mathcal{L}^2(\Omega, H)}:=\int_\Omega \langle X(\omega), Y(\omega)\rangle_H dP(\omega)
\end{equation}

\begin{definition}[uncorrelated, covariance operator]
  Let $H$ be a Hilbert space. A linear operator $\mathcal{C}:H\rightarrow H$ is the covariance of $H$-valued random variables $X$ and $Y$ if 
  \begin{equation}
      \langle\mathcal{C}_{XY}\phi, \psi\rangle_H = \operatorname{Cov}\left(\langle X, \phi\rangle_H, \langle Y, \psi\rangle_H\right), \forall \phi, \psi \in H
  \end{equation}
\end{definition}

\begin{example}[$H = \mathbb{R}^d$]
  In finite dimensional case $H = \mathbb{R}^d$, the covariance matrix conincides with the covariance operator. 
  \begin{equation}
      \begin{aligned}
          &\operatorname{Cov}\left(\langle X, \phi\rangle, \langle Y, \psi\rangle\right) 
          = \operatorname{Cov}\left(\phi^T X, \psi^T Y\right)\\
          =&\mathbf{E}\left[\phi^T(X-\mu_X)(Y-\mu_Y)^T\psi\right] 
          = \phi^T\mathbf{E}\left[(X-\mu_X)(Y-\mu_Y)^T\right]\psi\\
          =&\phi^T Cov(X, Y)\psi = \langle C_{XY}\phi, \psi\rangle
      \end{aligned}
  \end{equation}
\end{example}

\begin{example}[$H = L^2(D)$]
  When $X = Y$ noted as $u(x, \omega)\in H=L^2(D)$, the covariance operator:
  The covariance operator $\mathcal{C}$ can be defined as:
  \begin{equation}
  \langle\mathcal{C}_u\phi, \psi\rangle_{L^2(D)} = \int_D\int_D Cov(u(x), u(y)) \phi(x) \psi(y) dx dy
  \end{equation}
  So that for $\forall x\in D$,
  \begin{equation}
  (\mathcal{C}_u\phi)(x) = \int_D Cov(u(x), u(y)) \phi(y) dy
  \end{equation}
  That is any $L^2(D)$-valued random variable $u(x)$ can defines a R.F. with $\mu(x)$ and $C(x, y)$ equal to the integral kernel of $\mathcal{C}$.
\end{example}

\begin{theorem}\label{thmtraceclass}
  For random field $u(x,\omega)$ with the covariance operator $\mathcal{C}$, suppose $\mathcal{C}$ is trace class with eigenpairs $(\lambda_i, \phi_i)$, 
  then the second moment of $u(x, \omega)$ is given by:
  \begin{equation}
    \mathbf{E}[\|u(x, \omega)\|^2_H] = \sum_{i=1}^{\infty} \lambda_i
  \end{equation}
\end{theorem}

\begin{definition}[H-valued Gaussian random variable]
  Let $H$ be a Hilbert space. An H-valued random variable $u(x, \omega)$ is Gaussian if 
  $\langle u(x, \omega), \phi\rangle_H$ is a real-valued Gaussian random variable for all $\phi \in H$.
  Here the real-valued Gaussian Random Variable is defined as:
  \begin{equation}
    \langle u, \phi\rangle_H \sim N(\langle \mu, \phi\rangle_H, \langle \mathcal{C}_u\phi, \phi\rangle_H)
  \end{equation}
  This actually defines the Gaussian Measure on $H$: $u\sim N(\mu, \mathcal{C}_u):= m$, where $m$ is called Gaussian Measure. 
  The covariance operator of $u$ is the symmetric, positive-definite operator $\mathcal{C}_u : H \rightarrow H$.
\end{definition}

Since we consider infinite dimensional case, unlike finite dimensions, not all translations preserve the measure. 
We need to cansider those directions in $H$ along which translating a Gaussian measure does not change its essential nature (i.e., keeps it equivalent). 
\begin{definition}[Cameron-Martin Space]
The Cameron-Martin space $U$ is defined as:
\begin{equation}
  U:=\{h\in H| m_h\ll m\},m_h(A)=T_h^\#m(A):= m(A-h),\forall A \in \mathcal{B}(H)
\end{equation} 
where $T_h$ is the translation operator: $T_h(u)=u + h$, $T_h^\#m$ is the push-forward measure of $m$ by $T_h$.
\end{definition}

\begin{theorem}[Cameron-Martin Theorem]
  For $m=N(0, \mathcal{C})$, we have:
  \begin{equation}
    m_h\sim m \text{ if and only if } h\in \mathcal{C}^\frac{1}{2}H:=U
  \end{equation}
  Since $\mathcal{C}^{-1}$ is unbounded, $U$ is a proper subspace of $U = \mathcal{C}^{1/2}H$.
  The inner product is defined as:
\begin{equation}
  \langle \phi, \psi\rangle_U = \langle \mathcal{C}^{-1/2}\phi, \mathcal{C}^{-1/2}\psi\rangle_H
\end{equation}
\end{theorem}

Unlike the finite dimensional case, there is no natural Brownian Motion process in infinite dimensions. 
So for any Hilbert space $H$, we need to define the Brownian Motion process on $H$ by using the Cameron-Martin space $U$.
So we first define the $\mathcal{C}$-Weiner process $W_t$ on $H$ as:
\begin{definition}[$\mathcal{C}$-Weiner Process]
  A H-valued process $W_t$ is called $\mathcal{C}$-Weiner process if:
  \begin{itemize}
    \item $W_0 = 0$
    \item $W_t$ has continuous trajectories
    \item $W_t$ has independent increments
    \item $W_t$ has Gaussian increments: $m(W_t - W_s) = N(0, (t-s)\mathcal{C})$
  \end{itemize}
\end{definition}
Then we can define the $\mathcal{C}$-Weiner process $W_t$ on $U$ as:
\begin{equation}
  W^U_t = \mathcal{C}^{1/2}W^H_t
\end{equation}


\section{For Stationary RF on $\mathbb{R}^d$}
First we define stationary random field on $\mathbb{R}^d$ as:
\begin{definition}[Stationary Random Field]
  A second-order random field ${u(x): x\in D}$ is called stationary if the mean is constant and covariance function 
  depends only on the difference $x-y$, i.e. $\mu(x) = \mu,\ C(x, y) = C(x-y)$.
\end{definition}
Then we can define the covariance operator $\mathcal{C}$ as:
\begin{equation}
  \mathcal{C}\phi = \int_{\mathbb{R}^d} C(x-y)\phi(y)dy
\end{equation}
We find that it is actually the convolution operator of $C(x)$ with $\phi(x)$.

Stationary random fields have some beautiful properties.
\begin{theorem}[Wiener-Khinchin Theorem]
    There exists a stationary random field ${u(x): x\in D}$ with mean $\mu$ and covariance function $c(x)$ that is mean square continuous if and only if 
    the function $c(x): \mathbb{R}^d\rightarrow \mathbb{R}$ is such that 
    \begin{equation}
        c(x) = \frac{1}{(2\pi)^{d}}\int_{\mathbb{R}^d} e^{ik \cdot x}dF(k)=\frac{1}{(2\pi)^{d}}\int_{\mathbb{R}^d} e^{ik \cdot x}S(k)dk = \left(\mathcal{F}^{-1}S\right)(x)
    \end{equation}
    where $F(k)$ is some measure on $\mathbb{R}^d$ called spectral distribution and $\hat{S}(x)$ is the Fourier transform of $S(k)$, 
	called spectral density.
    Reversely, $S(k) = \left(\mathcal{F}c\right)(k) = \hat{c}(k)$.
    If $S(k)$ is non-negative and integrable, then $c(x)$ is a valid covariance function.
\end{theorem}

\begin{theorem}[Spectral Density of Random Field]\label{spectral_density_random_field}
	Assume $u(x)$ has zero mean, then 
\begin{equation}\label{spectraldensity}
	S_u(k)=\frac{1}{(2\pi)^{d}}\mathbb{E}[|\hat{u}(k)|^2]
\end{equation}
\end{theorem}



% \begin{equation}
% 	\int_{\mathbb{R}^d} \int_{\mathbb{R}^d} e^{-ik(x-y)}c(x-y)dxdy
% \end{equation}

By defining the pseudo-differential operators, the class of SPDEs is defined by:
\begin{equation}
	\mathcal{L}_gu = W, \mathcal{L}_g = \mathcal{F}^{-1}g\mathcal{F}
\end{equation}
where $g:\mathbb{R}^d\rightarrow \mathbb{C}$ must be a sufficiently regular and Hermitian-symmetric function, that is it must satisfy: $g(k) = \overline{g(-k)}$, $\overline{\cdot}$ denotes the complex conjugate.
So if we have $\mathcal{L}_gu = W$, then:
\begin{equation}
	u=\mathcal{L}_{\frac{1}{g}}W
\end{equation}

\begin{theorem}
	The spectral density of $\mathcal{L}_gu$ and of $u$ are related by:
	\begin{equation}
		S_{\mathcal{L}_gu}(k) = \left|g(k)\right|^2S_u(k)
	\end{equation}
	Generally, if 
	\begin{equation}\label{SPDEGeRF}
		\mathcal{L}_gu = w
	\end{equation}
	where $w$ is a GeRF source term, then $S_w(k) = \left|g(k)\right|^2S_u(k)$.
	Therefore, when $w = W$, $S_u =\frac{1}{(2\pi)^{d}\left|g(k)\right|^2}\mathbb{E}[|\hat{W}(k)|^2]=\frac{1}{\left|g(k)\right|^2}$. 
  Then, 
\begin{equation}
	u(x) =\mathcal{L}_{\frac{1}{g}}w(x) = \mathcal{L}_{\sqrt{\frac{S_u}{S_w}}}w(x)
\end{equation}
\end{theorem}

Then consider the exitence of the function.
\begin{theorem}
	Let $w(x)$ be a real stationary GeRF over $\mathbb{R}^d$, and let $g:\mathbb{R}^d\rightarrow \mathbb{C}$ be a symbol function. 
	Then for (\ref{SPDEGeRF}), there exists a unique stationary solution $u(x)$ if and only if:
	there exists $N\in \mathbb{N}$ s.t. 
	\begin{equation}
		\int_{\mathbb{R}^d}\frac{dS_w(k)}{\left|g(k)\right|^2(1+\|k\|^2)^N} < \infty
	\end{equation}
	and 
	\begin{equation}
		S_u(k) = \left|g(k)\right|^{-2}S_w(k)
	\end{equation}
	Moreover, $S_u(k)$ is unique if and only if $\left|g\right|>0$.
\end{theorem}

% Inspired by SPDE approach: 
% \begin{theorem}
%     For $\forall u\in U:=L^2(D\times \Omega)$, a stationary random field with covariance function $c(x)$, $\exists L \in \mathcal{L}(U)$ s.t.
% 	$\mathcal{L}u = W$, where $W$ is a spatial Gaussian white noise with unit variance. 
% \end{theorem}

% Hence we can use a DNN-based model as the surrogate operator $\mathcal{N}$ of $L$. Here we use the Fourier transform to encode the solution $u$ to Fourier space: 
% \begin{equation}
%     \mathcal{F}(Lu)(k)=\hat{L}(k)\hat{u}(k) = \hat{W}\Rightarrow u(x) = \mathcal{F}^{-1}\left(\frac{\hat{W}}{\hat{L}}\right)(x)=\mathcal{F}^{-1}\left(\hat{\mathcal{N}}\hat{W}\right)(x)
% \end{equation}
Hence the key is the symbol function $g(k)$.
The following theorem shows that solutions of SPDEs with White Noise source term is the starting point of more general solutions, when the source term can be any stationary GeRF.
\begin{theorem}\label{uniquenessandexistence}
	Let $w(x)$ be a real stationary GeRF over $\mathbb{R}^d$ with covariance distribution $C_w(x)$. 
	Let $g$ be a symbol function over $\mathbb{R}^d$ such that $\frac{1}{g}$ is smooth with polynomially bounded derivatives of all orders. 
	Then, there exists a unique stationary solution to (\ref{SPDEGeRF}) and its covariance distribution is given by
\begin{equation}
	C_u(x) = C_u^W * C_w(x)
\end{equation}
where $C_u^W$ is the covariance function of the solution to the SPDE with White Noise source term.
\end{theorem}
\begin{proof}
	The proof is straightforward by using Weiner-Khinchin theorem.
\end{proof}
For any precision operator which is a polynomial in the Laplacian, $Q = p(-\Delta)$, such as the Matern operator with $\nu \in \mathbb{N}$, 
this results in a polynomial $F(Q) = p(\|k\|^2)$.
\subsection{Matern Field}
The important relationship that we will make use of is that a Gaussian field $u(x)$ with the Matern covariance is 
a solution to the linear fractional stochastic partial differential equation (SPDE):
\begin{equation}\label{SPDE}
	\mathcal{L}^{\alpha/2}u(x) = (\kappa^2 - \Delta)^{\alpha/2} u(x) = W(x), \qquad x\in D\in \mathbb{R}^d, \alpha=\nu + d/2, \kappa>0, \nu>0,
\end{equation}
where $\nu = \alpha - d/2, \rho = \frac{\sqrt{2\nu}}{\kappa}$ is the range parameter, $\Delta$ is the Laplacian operator, $W(x)$ is a spatial Gaussian white noise with unit variance.
We will name any solution to Equ (\ref{SPDE}) a Matern field in the following. 

\begin{theorem}[Spectral Solution of Matern Field]\label{spectral_solution_matern}
	The solution of u solved by Equ (\ref{SPDE}) is given by:
	\begin{equation}
		u(x) = \mathcal{F}^{-1}\left[\frac{\hat{W}(k)}{(\kappa^2 + \|k\|^2)^{\alpha/2}}\right](x)
	\end{equation}
	where $\mathcal{F}$ is defined in (\ref{FourierTransform}).
	And the covariance function of u is given by:
	\begin{equation}
		c(x) = \frac{\sigma^2}{2^{\nu -1}\Gamma(\nu)}(\kappa \|x\|)^\nu K_\nu (\kappa \|x\|)
	\end{equation}
	where $\nu = \alpha - d/2, \rho = \frac{\sqrt{2\nu}}{\kappa}, \sigma^2 = \frac{\Gamma(\nu)}{(4\pi)^{d/2}\kappa^{2\nu}\Gamma(\alpha) }$
\end{theorem}


% By Wiener-Khinchin theorem, we have known that: given a stationary covariance function $c(x)$ 
% that is mean square continuous, we can always have a spectral density $S(k)$. Then the problem comes to:
% if exists an operator $\mathcal{L}$ s.t. 

Weiner-Khinchin theorem + Spectral Theorem.
\subsection{Generalized Matern Field}
Consider the following SPDE:
\begin{equation}\label{GWM}
	(\kappa^2 +(- \Delta)^{\gamma})^{\alpha/2}u(x)=\mathcal{F}^{-1}\left((\kappa^2 + \|k\|^{2\gamma})^{\alpha/2}\mathcal{F}u\right)(x) = W(x), \quad x\in D\in \mathbb{R}^d
\end{equation}
Hence the solution is:
\begin{equation}
	u(x)=\mathcal{F}^{-1}\left[\frac{\hat{W}(k)}{(\kappa^2 + \|k\|^{2\gamma})^{\alpha/2}}\right](x)
\end{equation}
Therefore the spectral density is:
\begin{equation}
	S_u(k)=\frac{1}{(\kappa^2 + \|k\|^{2\gamma})^{\alpha}}
\end{equation}
So when $\gamma = 1$, it becomes the Matern Field. Since  the spectral density $S_u(k)\in L^2(\mathbb{R}^d)$ if and only if $\alpha \gamma > \frac{d}{2}$.

Generally, we can define the pseudo-differential operator through symbol function $g(k)$.
\section{Spatial-Temporal General Random Field on $\mathbb{R}^d\times (0, T)$}
\subsection{Stein Model}
Proposed in \cite{stein2005space}, we define the spatial-temporal white noise $\mathcal{W}(x,t)$ as Gaussian noise that is white in time but correlated in space.
\begin{equation}\label{SteinModel}
	\left(b(s^2-\frac{d}{dt^2})^\beta + a(\kappa^2-\Delta)^\alpha\right)^{\nu / 2}u(x,t) = W,\qquad (x, t)\in D\times (0, T)
\end{equation}
Consider when $D = \mathbb{R}^d$, where the space-time spectral density of the stationary solution is given by:
\begin{equation}
	S_u(k_s, k_t) = \frac{1}{(b(s^2 + k_t^2)^\beta + a(\kappa^2 + k_s^2)^\alpha)^\nu}
\end{equation}
We note the spatio-temporal symbol function as:
\begin{equation}
	g(k_s, k_t): (k_s, k_t)\rightarrow (b(s^2 + k_t^2)^\beta + a(\kappa^2 + k_s^2)^\alpha)^{\nu/2}
\end{equation}
When $\alpha, \beta, \nu$ are positive and $\frac{d}{\alpha\nu}+\frac{1}{\beta\nu} = 2$, 
\cite{stein2005space} shows that the spectral density is finite and the corresponding random field is mean square continuous.

\begin{theorem}
	When $\kappa, s, a, b>0$ and $\alpha, \beta, \nu$ are not null, $g(k_s, k_t)$ satisfies Thm \ref{uniquenessandexistence}, 
then for any stationary GeRF $X$, the SPDE:
\begin{equation}
	\left(b(s^2-\frac{d}{dt^2})^\beta + a(\kappa^2-\Delta)^\alpha\right)^{\nu / 2}U(x,t) = X(x,t)
\end{equation}
has a unique stationary solution $U(x,t)$ with covariance function:
\begin{equation}
	C_U(x, y, t, s) = C_U^W*C^X(x-y, t-s)
\end{equation}
\end{theorem}

\subsection{Evolution Equations Model}
Here we consider the following model:
\begin{equation}
	\frac{\partial^\beta u}{\partial t^\beta} + \mathcal{L}_gu = w(x,t)
\end{equation}
where $\mathcal{L}_g$ is a pseudo-differential operator with symbol $g(k)$ and $w(x,t)$ is a stationary spatio-temporal GeRF.
\begin{equation}
	g(k_s, k_t) = (ik_t)^\beta + g(k_s)
\end{equation}

\subsection{Advection-Diffusion SPDE}
This is poeposed in \cite{sigrist2015stochastic}. The equation is given by:
\begin{equation}
	\left[\frac{\partial}{\partial t} - \nabla \cdot (\Sigma \nabla)+\mu \nabla + C\right]u(x,t) = w(x,t)
\end{equation}
where $\Sigma$ is the diffusion matrix, $\mu$ is the advection velocity, $C$ is the drift coefficient.
Here we set the diffusion matrix as:
\begin{equation}
	\Sigma = \frac{1}{\rho^2}\begin{pmatrix}
		\cos\theta & \sin\theta \\
		-\gamma\sin\theta & \gamma\cos\theta
	\end{pmatrix}^T\begin{pmatrix}
		\cos\theta & \sin\theta \\
		-\gamma\sin\theta & \gamma\cos\theta
	\end{pmatrix}
\end{equation}
where $\rho$ is the correlation length and $\gamma$ is the anisotropy ratio, $\theta\in [0, \pi/2)$. With $\gamma = 1$, it becomes isotropic.
Similarly the spectral density is given by:
\begin{equation}
	\begin{aligned}
		S_u(k_s, k_t) &= \frac{S_w(k_s, k_t)}{\left|i(k_t + \mu k_s) + (C + k_s^T\Sigma k_s)\right|^2}\\
		&= \frac{S_w(k_s, k_t)}{(k_t + \mu k_s)^2 + (C + k_s^T\Sigma k_s)^2}\\
		&= \frac{S_w(k_s, k_t)}{\left|g_u\right|^2}\\
	\end{aligned}
\end{equation}
By Wiener-Khinchin theorem, the covariance function is given by:
\begin{equation}
	C_u(x, t) = \frac{1}{(2\pi)^{d}}\int S_w\frac{e^{-i\mu k_s t-(k_s^T\Sigma k_s + C)|t|}}{2(k_s^T\Sigma k_s + C)}e^{ik_s x}dk_s
\end{equation}
Specifically, when $\mu = 0, \Sigma = 0$, the covariance function is given by:
\begin{equation}
	C_u(x, t) = \frac{e^{-C|t|}}{2C}C_w(x, t)
\end{equation}
However $\mu(x)$ may not be constant.

\subsection{Generic class of non-stationary models}
Similar to ADSPDE, we consider:
\begin{equation}
	\frac{\partial u}{\partial t} + \left[ - \nabla \cdot (\Sigma(x, t) \nabla)+\mu(x, t)\cdot \nabla + \kappa^2(x, t)\right]^{\alpha/2}u(x,t) = w(x,t)
\end{equation}
where $\mu, \Sigma, \kappa$ are functions of $x, t$, and $w(x, t)$ is a GeRF driven by Equ (\ref{SPDE}).


\section{Experiments}
We must pay particular attention to the change in normalization factors when moving from the continuous to the discrete setting. Specifically:
\begin{itemize}
	\item Regarding the simulation of white noise:
	We use $N(0,1/h)$ to approximate the characteristic variance $\delta(x-y)$ of continuous white noise. 
	However, when calculating the spectral density (\ref{Suk}), we need to carefully incorporate appropriate normalization factors:
	\begin{equation}
		S_u(k) = \frac{1}{N\frac{1}{h}}\frac{\mathbb{E}\left[\left|DFT(W)(k)\right|^2\right]}{(\kappa^2 + \|k\|^2)^{\alpha}} 
	\end{equation}
	\item Fourier transform definitions:
	Special attention must be given to the definition of the Fourier transform. Different normalization conventions will result in distinct normalization factors and, consequently, different results.
\end{itemize}
First we test the Matern Covariance function:
\begin{figure}[h]
    \centering
    \includegraphics[width=0.5\linewidth]{pics/1D_Empirical_vs_Theoretical_Covariance_2048_100.0_2_1.png}
    \caption{1D stationary covariance function}
    \label{1d_matern_covariance}
\end{figure}

\subsection{1D random fields}
Here are some examples of 1D random fields computed by classical methods in Fig \ref{1drf}.
\begin{figure}[H]
    \centering
\begin{minipage}[t]{0.3\textwidth}  % 宽度可自定义,例如 0.45\textwidth
    \centering
    \includegraphics[width=\textwidth]{./pics/1D_RF_4096_100.0_0.5_1.png}  % 图片路径及格式
    \caption*{$\nu = 0.5, \kappa = 1$}
  \end{minipage}
  \begin{minipage}[t]{0.3\textwidth} 
    \centering
    \includegraphics[width=\textwidth]{./pics/1D_RF_4096_100.0_1_1.png} 
    \caption*{$\nu = 1, \kappa = 1$}
    % \label{polymesh}
  \end{minipage}
  \begin{minipage}[t]{0.3\textwidth}
    \centering
    \includegraphics[width=\textwidth]{./pics/1D_RF_4096_100.0_1.5_1.png}  % 图片路径及格式
    \caption*{$\nu = 1.5, \kappa = 1$}
    % \label{prediction}
  \end{minipage}

\begin{minipage}[t]{0.3\textwidth}  % 宽度可自定义,例如 0.45\textwidth
    \centering
    \includegraphics[width=\textwidth]{./pics/1D_RF_4096_100.0_2_1.png}  % 图片路径及格式
    \caption*{$\nu = 2, \kappa = 1$}
  \end{minipage}
  \begin{minipage}[t]{0.3\textwidth} 
    \centering
    \includegraphics[width=\textwidth]{./pics/1D_RF_4096_100.0_6_1.png} 
    \caption*{$\nu = 6, \kappa = 1$}
    % \label{polymesh}
  \end{minipage}
  \begin{minipage}[t]{0.3\textwidth}
    \centering
    \includegraphics[width=\textwidth]{./pics/1D_RF_4096_100.0_10_1.png}  % 图片路径及格式
    \caption*{$\nu = 10, \kappa = 1$}
    % \label{prediction}
  \end{minipage}

    \caption{1D Random Field (Normalized)}
    \label{1drf}
\end{figure}

Use MLP to learn the spectral density of the random field, and then generate the random field by inverse Fourier transform.
\begin{figure}[H]
    \centering
\begin{minipage}[t]{0.3\textwidth}  % 宽度可自定义,例如 0.45\textwidth
    \centering
    \includegraphics[width=\textwidth]{./pics/1D_NeuralRF_4096_100.0_0.5_1.png}  % 图片路径及格式
    \caption*{$\nu = 0.5, \kappa = 1$}
  \end{minipage}
  \begin{minipage}[t]{0.3\textwidth} 
    \centering
    \includegraphics[width=\textwidth]{./pics/1D_NeuralRF_4096_100.0_2_1.png} 
    \caption*{$\nu = 2, \kappa = 1$}
    % \label{polymesh}
  \end{minipage}
  \begin{minipage}[t]{0.3\textwidth}
    \centering
    \includegraphics[width=\textwidth]{./pics/1D_NeuralRF_4096_100.0_10_1.png}  % 图片路径及格式
    \caption*{$\nu = 10, \kappa = 1$}
    % \label{prediction}
  \end{minipage}

    \caption{1D Neural Random Field}
    \label{1dnrf}
\end{figure}


Assume discrete samples of the stochastic simulator $\mathcal{M}(x, \omega)$ are given as:
\begin{equation}
	\mathcal{T}_r :=\left\{\left(x_i^r, \mathcal{M}(x_i^r, \omega^r)\right): i=1,2,\cdots,N^r\right\}, \quad r=1,2,\cdots,R.
\end{equation}
in the form of discrete evaluations of the stochastic simulator on R trajectories, where for
every r, $\{x_i^r\}_{i=1}^{N^r}$ is an i.i.d. sample from the input distribution $\mu$, and $\mathcal{M}(x_i^r, \omega^r)$ is the output of the stochastic simulator.
Here we assume for notational simplicity that $N^r = N$ for all $r$.

That is we have $R$ samples $u(x)$. By the definition of spectral density, we can have the estimation of $S_u(k)$:
\begin{equation}
    S_u(k) = \frac{1}{(2\pi)^d}\frac{1}{R}\sum_{r=1}^R\left(\left\|(\mathcal{F}u_r)(k)\right\|^2\right)
\end{equation}
For the above example, the solution can be rewritten as:
\begin{equation}
    u(x) = \mathcal{F}^{-1}\left[\hat{W}(k)(S_u(k))^{1/2}\right](x)
\end{equation}

\begin{figure}[H]
    \centering
    \includegraphics[width=0.5\linewidth]{pics/1D_DataDrivenRF_4096_100.0_2_1.png}
    \caption{1D Data-Driven Random Field}
    \label{DataDrivenRF}
\end{figure}



\subsection{2D random fields}
2D is similar. Neural 2D random field is vertified.
\begin{figure}[H]
    \centering
\begin{minipage}[t]{0.3\textwidth}  % 宽度可自定义,例如 0.45\textwidth
    \centering
    \includegraphics[width=\textwidth]{./pics/2D_RF_256_10.0_0.5_1.png}  % 图片路径及格式
    \caption*{$\nu=0.5, \kappa=1$}
  \end{minipage}
  \begin{minipage}[t]{0.3\textwidth} 
    \centering
    \includegraphics[width=\textwidth]{./pics/2D_RF_256_10.0_1_1.png} 
    \caption*{$\nu=1, \kappa=1$}
    % \label{polymesh}
  \end{minipage}
  \begin{minipage}[t]{0.3\textwidth}
    \centering
    \includegraphics[width=\textwidth]{./pics/2D_RF_256_10.0_2_1.png}  % 图片路径及格式
    \caption*{$\nu=2, \kappa=1$}
    % \label{prediction}
  \end{minipage}

\begin{minipage}[t]{0.3\textwidth}  % 宽度可自定义,例如 0.45\textwidth
    \centering
    \includegraphics[width=\textwidth]{./pics/2D_RF_256_10.0_3_1.png}  % 图片路径及格式
    \caption*{$\nu=3, \kappa=1$}
  \end{minipage}
  \begin{minipage}[t]{0.3\textwidth} 
    \centering
    \includegraphics[width=\textwidth]{./pics/2D_RF_256_10.0_6_1.png} 
    \caption*{$\nu=6, \kappa=1$}
    % \label{polymesh}
  \end{minipage}
  \begin{minipage}[t]{0.3\textwidth}
    \centering
    \includegraphics[width=\textwidth]{./pics/2D_RF_256_10.0_10_1.png}  % 图片路径及格式
    \caption*{$\nu=10, \kappa=1$}
    % \label{prediction}
  \end{minipage}

    \caption{2D Random Field}
    \label{2drf}
\end{figure}

\begin{figure}[H]
    \centering
\begin{minipage}[t]{0.3\textwidth}  % 宽度可自定义,例如 0.45\textwidth
    \centering
    \includegraphics[width=\textwidth]{./pics/2D_NeuralRF_256_10.0_0.5_1.png}  % 图片路径及格式
    \caption*{$\nu=0.5, \kappa=1$}
  \end{minipage}
  \begin{minipage}[t]{0.3\textwidth} 
    \centering
    \includegraphics[width=\textwidth]{./pics/2D_NeuralRF_256_10.0_1_1.png} 
    \caption*{$\nu=1, \kappa=1$}
    % \label{polymesh}
  \end{minipage}
  \begin{minipage}[t]{0.3\textwidth}
    \centering
    \includegraphics[width=\textwidth]{./pics/2D_NeuralRF_256_10.0_2_1.png}  % 图片路径及格式
    \caption*{$\nu=2, \kappa=1$}
    % \label{prediction}
  \end{minipage}

\begin{minipage}[t]{0.3\textwidth}  % 宽度可自定义,例如 0.45\textwidth
    \centering
    \includegraphics[width=\textwidth]{./pics/2D_NeuralRF_256_10.0_3_1.png}  % 图片路径及格式
    \caption*{$\nu=3, \kappa=1$}
  \end{minipage}
  \begin{minipage}[t]{0.3\textwidth} 
    \centering
    \includegraphics[width=\textwidth]{./pics/2D_NeuralRF_256_10.0_6_1.png} 
    \caption*{$\nu=6, \kappa=1$}
    % \label{polymesh}
  \end{minipage}
  \begin{minipage}[t]{0.3\textwidth}
    \centering
    \includegraphics[width=\textwidth]{./pics/2D_NeuralRF_256_10.0_10_1.png}  % 图片路径及格式
    \caption*{$\nu=10, \kappa=1$}
    % \label{prediction}
  \end{minipage}

    \caption{2D Neural Random Field}
    \label{2dnrf}
\end{figure}



\section{Conclusion}
Q1: For general SPDE, given the equation, how to simulate the random field? 
Or inversely, given the random field data, how to infer the equation?

Q2: What is the relationship between the equation and the covariance function? 
For $\forall \mathcal{M}(x, \omega)$, can we find a differential operator $\mathcal{L}$ and a noise $\mathcal{W}$ s.t. $\mathcal{L}\mathcal{M} =\mathcal{W}$?

Q3: Generally speaking, given the covariance function of $\mathcal{M}(x, \omega)$, 
how can we simulate the random field with DNN?

Consider:
\begin{itemize}
	\item Generative Model: GAN, VAE, Flow-based model
	\item Neural Operator: DeepONet, FNO, etc.
\end{itemize}

\newpage
\bibliographystyle{plain}
\bibliography{references}

\newpage
\appendix
\section{Fourier Transform}\label{FourierTransform}
\begin{definition}
	Define the Fourier transform of $u(x)$ as:
	\begin{equation}\left\{
		\begin{aligned}
			& \mathcal{F}u(k) = \int_{\mathbb{R}^d} u(x) e^{-ikx} dx,\\
			& \mathcal{F}^{-1}u(x) = \frac{1}{(2\pi)^{d}}\int_{\mathbb{R}^d} u(k) e^{ikx} dk.
		\end{aligned}\right.
	\end{equation}
\end{definition}
\section{Polynomial Chaos Expansion}\label{PCE}
A spectral expansion in $L_{\mu}(D)$ is called chasos expansion. By defining the inner product in $L_{\mu}(D)$ as:
\begin{equation}
	\langle f, g \rangle_{\mu} = \int_D f(x) g(x) d\mu(x),
\end{equation}
the space of $L_\mu(D)$ is a Hilbert space.
\begin{theorem}
	Let $\{\phi_i(x)\}_{i=1}^{\infty}$ be an orthonormal basis of $L_\mu(D)$, i.e.
	\begin{equation}
		\begin{aligned}
			&1. \int_D \phi_i(x) \phi_j(x) d\mu(x) = \delta_{ij},\\
			&2. {\phi_i(x)} \text{ is dense in } L_\mu(D).
		\end{aligned}
	\end{equation}
	Then the chasos expansion of $\mathcal{M}(x, \omega)\in L_\mu(D)$ is given by:
	\begin{equation}
		\mathcal{M}(x, \omega) = \sum_{i=1}^{\infty} c_i \phi_i(x),
	\end{equation}
	where $c_i(x)=\langle \mathcal{M}, \phi_i\rangle_{\mu}$ are the coefficients of the expansion.
\end{theorem}
	

\section{Karhunen-Loeve Expansion}\label{KLE}
Let $\mu(x)$ be the mean function and $C(x, y)$ be the covariance function of $\mathcal{M}(x, \omega)$. 
Assume that $D$ is bounded, $C(x, y)$ is continuous and $\mathcal{M}(x, \cdot)$ has finite variables for all $x\in D$.


\begin{theorem}\label{KL-expansion}
    The Karhunen-Loeve expansion of $\mathcal{M}(x, \omega)$ is given by:
    \begin{equation}
        \mathcal{M}(x, \omega) = \mu(x) + \sum_{i=1}^{\infty} \lambda_i \phi_i(x) \xi_i(\omega),
    \end{equation}
	where $\lambda_i$ and $\phi_i(x)$ are the eigenvalues and eigenfunctions of the covariance operator $\mathcal{C}$:
\begin{equation}
	(\mathcal{C}\phi_i)(x) =\int_D Cov(\mathcal{M}(x, \omega), \mathcal{M}(y, \omega)) \phi_i(y) dy= \int_D C(x, y) \phi_i(y) dy = \lambda_i \phi_i(x).
\end{equation}
where $C(x,y), x,y\in D$ is the covariance function of $\mathcal{M}(x, \omega)$. The KL-random variables $\xi_i(\omega)$ are the result of the projection of $\mathcal{M}(x, \omega)$ onto the eigenfunctions $\phi_i(x)$:
\begin{equation}
	\xi_i(\omega) = \frac{1}{\sqrt{\lambda_i}}\int_D (\mathcal{M}(x, \omega)-\mu(x)) \phi_i(x) dx.
\end{equation}
\end{theorem}
Note that both $\{\phi_i(x)\}_{i=1}^{\infty}$ and $\{\xi_i(\omega)\}_{i=1}^{\infty}$ are orthonormal bases, 
one capturing the “spatial” variation  of $\mathcal{M}(x, \omega)$ over $D$ (in terms of $x$), 
the other capturing the stochastic variation of $\mathcal{M}(x, \omega)$ (in terms of $\omega$).

\begin{theorem}[Mercer's theorem]
	The covariance function $C(x, y)$ of $\mathcal{M}(x, \omega)$ can be expressed as:
\begin{equation}
	C(x, y) = \sum_{i=1}^{\infty} \lambda_i \phi_i(x) \phi_i(y).
\end{equation}
It follows that the average variance of the random field over the domain $D$ is equal to $\sum_{i=1}^{\infty} \lambda_i$.
\end{theorem}

In most cases, the integral eigenvalue problem in Equ (\ref{KL-expansion}) is difficult: analytically solved and high-dimensional.

\section{Some proofs}
\begin{proof}[Proof of Thm \ref{thmtraceclass}]
\begin{equation}
  \begin{aligned}
    \mathbb{E}\left[\|u(x, \omega)\|^2_H\right] &= \mathbb{E} \left[\sum_{i=1}^\infty \langle u, \phi_i\rangle_H^2\right]\\
    &= \sum_{i=1}^\infty \mathbb{E} \left[\langle u, \phi_i\rangle_H^2\right]\\
    &= \sum_{i=1}^\infty \langle \mathcal{C}_u  \phi_i, \phi_i\rangle_H= \sum_{i=1}^\infty \lambda_i
  \end{aligned}
\end{equation}
\end{proof}

\begin{proof}[Proof of Thm \ref{spectral_density_random_field}]
    \begin{equation}
    \begin{aligned}
      S_u(k)&=\left(\mathcal{F}c\right)(k)= \int_{\mathbb{R}^d} e^{-ik h}c(h) dh\\
      &= \int_{\mathbb{R}^d} e^{-ik(x+h - x)}\mathbb{E}\left[u(x+h)u(x)\right] dh\\
      &= \mathbb{E}\left[\int_{\mathbb{R}^d} e^{-ik(x+h)}u(x+h)e^{ikx}u(x) dh\right]\\
      &= \frac{1}{(2\pi)^{d}}\mathbb{E}\left[\left|(\mathcal{F}u)(k)\right|^2\right]
    \end{aligned}
  \end{equation}
\end{proof}

\begin{proof}[Proof of Thm \ref{spectral_solution_matern}]
	Do Fourier transform on Equ (\ref{SPDE}):
\begin{equation}
	\left\{\mathcal{F}(\kappa^2 - \Delta)^{\alpha/2} u\right\}(k) = (\kappa^2 + \|k\|^2)^{\alpha/2}(\mathcal{F}u)(k)
\end{equation}
then we have 
% , and the marginal variance is
% \begin{equation}
% 	\sigma^2 = \frac{\Gamma(\nu)}{\Gamma(\nu+d/2) \kappa^{2\nu} (4\pi)^{d/2}}.
% \end{equation}
\begin{equation}\label{spectralsolution}
	(\mathcal{F}u)(k) = \hat{u}(k) = \frac{\hat{W}(k)}{(\kappa^2 + \|k\|^2)^{\alpha/2}} 
\end{equation}
Therefore, u can be written as:
\begin{equation}
	u(x) = \mathcal{F}^{-1}\left[\frac{\hat{W}(k)}{(\kappa^2 + \|k\|^2)^{\alpha/2}}\right]
\end{equation}
Then the stationary covariance function of u is given by:
\begin{equation}
	c(x) = Cov(u(x), u(0))
\end{equation}
By the definition of spectral density Equ (\ref{spectraldensity}) and Equ (\ref{spectralsolution}) we have:
\begin{equation}\label{Suk}
		S_u(k) = \frac{1}{(2\pi)^{d}}\frac{\mathbb{E}\left[\left|\hat{W}(k)\right|^2\right]}{(\kappa^2 + \|k\|^2)^{\alpha}} 
		= \frac{1}{(\kappa^2 + \|k\|^2)^{\alpha}} 
\end{equation}
Then we have the variance of u:
\begin{equation}
	c(0) = \frac{1}{(2\pi)^{d}}\int_{\mathbb{R}^d} S_u(k) dk = \frac{\Gamma(\nu)}{(4\pi)^{d/2}\kappa^{2\nu}\Gamma(\alpha) }:=\sigma^2
\end{equation}
	By Wiener-Khinchin theorem, we have:
\begin{equation}
	\begin{aligned}
		c(x) &= (\mathcal{F}^{-1}S_u)(x)=\mathcal{F}^{-1}\left[\frac{1}{(\kappa^2 + \|k\|^2)^{\alpha}}\right]\\
		&= \frac{\|x\|^\nu K_{\nu}(\kappa\|x\|)}{(4\pi)^{d/2}2^{\nu-1}\kappa^{\nu}\Gamma(\alpha)}\\
		& = \frac{\sigma^2}{2^{\nu -1}\Gamma(\nu)}(\kappa \|x\|)^\nu K_\nu (\kappa \|x\|)
	\end{aligned}
\end{equation}
\end{proof}
\begin{remark}
	To make $c(0) = 1$, we can multiple a constant factor $\sigma_1$ to $S_u(k)$:
	\begin{equation}
		\sigma_1^2 = \frac{\Gamma(\alpha) \kappa^{2\nu}(4\pi)^{d/2}}{\Gamma(\nu)}
	\end{equation}
	Then the corresponding function of u is:
	\begin{equation}
		u(x) = \mathcal{F}^{-1}\left[\frac{\sigma_1\hat{W}(k)}{(\kappa^2 + \|k\|^2)^{\alpha/2}}\right]
	\end{equation}
\end{remark}


\end{document}