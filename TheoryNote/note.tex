% \documentclass{ctexart}
\documentclass{article}
\usepackage{listings}                                           %插入代码
\usepackage{geometry}                                           %设置页面大小边距等
\usepackage{graphicx}                                           %插入图片
\usepackage{amssymb}                                            %为了用\mathbb
\usepackage{amsmath}                                            %数学方程的显示
\usepackage{listings}                                           %插入代码
\usepackage{fancyhdr}                                           %设置页眉页脚
\usepackage{lastpage}                                           %总页数
\usepackage{hyperref}                                           %引用网页
\usepackage{xcolor}
\usepackage{tikz}
\usepackage{float}
\usepackage{subcaption} 
\usepackage{mathrsfs}
\usepackage{amsthm}



\geometry{a4paper,left=2cm,right=2cm,top=2cm,bottom=2cm}        %一定要放在前面!
\pagestyle{fancy}                                               %设置页眉页脚
\lhead{Guanyu Chen}                                             %页眉左Fir                                        

\rhead{Stochastic Differential Equation} 
\cfoot{\thepage/\pageref{LastPage}}                             %当前页,记得调用前文提到的宏包
\lfoot{Zhejiang University}
\rfoot{College Of Integrated Circuits}
\renewcommand{\headrulewidth}{0.1mm}                            %页眉线宽,设为0可以去页眉线
\renewcommand{\footrulewidth}{0.1mm}                            %页脚线宽,设为0可以去页眉线
\setlength{\headwidth}{\textwidth}

\hypersetup{                                                    %设置网页链接颜色等
    colorlinks=true,                                            %链接将会有颜色,默认是红色
    linkcolor=blue,                                             %内部链接,那些由交叉引用生成的链接将会变为蓝色(blue)
    filecolor=magenta,                                          %链接到本地文件的链接将会变为洋红色(magenta)
    urlcolor=blue,                                              %链接到网站的链接将会变为蓝绿色(cyan)
}

\lstset{  
    basicstyle=\ttfamily,  
    keywordstyle=\color{blue},  
    language=Python,  
    numbers=left,  
    numberstyle=\tiny\color{gray},  
    frame=single,  
    breaklines=true  
}  

\newtheorem{theorem}{Theorem}
% \newtheorem{proof}{Proof}
\newtheorem{solution}{Solution:}
\newtheorem{remark}{Remark}
\newtheorem{definition}{Definition}
\newtheorem{algorithm}{Algorithm}
\newtheorem{lemma}{Lemma}
\newtheorem{example}{Example}
\newtheorem{problem}{Problem}

\title{About Stochastic Differential Equation}
\author{Guanyu Chen}
\date{\today}
\begin{document}
\maketitle
\tableofcontents
\newpage
\section{Fokker-Planck-Kolmogorov equation}
\begin{problem}
Assume we have a Stochastic Differential Equation like:
\begin{equation}\label{sde}
    dX_t = f(X_t, t)dt + G(X_t, t)dW_t
\end{equation}
where $X_t\in \mathbf{R}^d,f\in \mathcal{L}(\mathbf{R}^{d+1}, \mathbf{R}^d)$, and $W_t$ is m-dim Brownian Motion with diffusion matrix $Q$, 
$G(X_t, t)\in \mathcal{L}(\mathbf{R}^{m+1}, \mathbf{R}^d)$, with initial condition $X_0\sim p(X_0)$.
\end{problem}
\begin{definition}[Generator]
    The infinitesimal generator of a stochastic process $X(t)$ for function $\phi(x)$, i.e. $\phi(X_t)$ can be defined as
    \begin{equation}
        \mathcal{A} \phi(X_t)=\lim _{s \rightarrow 0^{+}} \frac{E[\phi(X(t+s)]-\phi(X(t))}{s}
    \end{equation}
    Where  $\phi$  is a suitable regular function.
\end{definition}
This leads to Dynkin's Formula very naturally.
\begin{theorem}[Dynkin's Formula]
    \begin{equation}
        E[f(X_t)]=f(X_0)+E\left[\int_0^t\mathcal{A}(f(X_s))ds\right]
    \end{equation}
\end{theorem}

\begin{theorem}
    If  $X(t)$  s.t. \ref{sde}, then the generator is given:
\begin{equation}
    \mathcal{A}(\cdot)=\sum_{i} \frac{\partial(\cdot)}{\partial x_{i}} f_{i}(X_t, t)+\frac{1}{2} \sum_{i, j}\left(\frac{\partial^{2}(\cdot)}{\partial x_{i} \partial x_{j}}\right)\left[G(X_t, t)Q G^{\top}(X_t, t)\right]_{i j}
\end{equation}
\end{theorem}
\begin{proof}
    See P119 of SDE by Oksendal.
\end{proof}

\begin{example}
    If $dX_t=dW_t$, then $\mathcal{A}=\frac{1}{2}\Delta$, where $\Delta$ is the Laplace operator.
\end{example}

\begin{definition}[Generalized Generator]
    For $\phi(x, t)$, i.e. $\phi(X_t, t)$, the generator can be defined as:
    \begin{equation}
        A_{t} \phi(x, t)=\lim _{s \rightarrow 0^{+}} \frac{E[\phi(X(t+s), t+s)]-\phi(X(t), t)}{s}
    \end{equation}
\end{definition}

\begin{theorem}
    Similarly if $X(t)$ s.t. \ref{sde}, then the generalized generator is given:
    \begin{equation}
        \mathcal{A}_{t}(\cdot)=\frac{\partial(\cdot)}{\partial t}+\sum_{i} \frac{\partial(\cdot)}{\partial x_{i}} f_{i}(X_t, t)+\frac{1}{2} \sum_{i, j}\left(\frac{\partial^{2}(\cdot)}{\partial x_{i} \partial x_{j}}\right)\left[G(X_t, t) Q G^{\top}(X_t, t)\right]_{i j}
    \end{equation}
\end{theorem}
We want to consider the density distribution of $X_t, P(x, t)$
\begin{theorem}[Fokken-Planck-Kolmogorov equation]
    The density function $P(x, t)$ of $X_t$ s.t. \ref{sde} solves the PDE:
    \begin{equation}
        \frac{\partial P(x, t)}{\partial t}=-\sum_{i} \frac{\partial}{\partial x_{i}}\left[f_{i}(x, t) p(x, t)\right]+\frac{1}{2} \sum_{i, j} \frac{\partial^{2}}{\partial x_{i} \partial x_{j}}\left[\left(G Q G^{\top}\right)_{i j} P(x, t)\right]
    \end{equation}
    The PDE is called FPK equation / forwand Kolmogorov equation.
\end{theorem}
\begin{proof}
    Consider the function $\phi(x)$, let $x=X_t$ and apply Ito's Formula:
    \begin{equation}
        \begin{aligned}
            d \phi & =\sum_{i} \frac{\partial \phi}{\partial x_{i}} d x_{i}+\frac{1}{2} \sum_{i, j}\left(\frac{\partial^{2} \phi}{\partial x_{i} \partial x_{j}}\right) d x_{i} d x_{j} \\
            & =\sum_{i} \frac{\partial \phi}{\partial x_{i}}\left(f_{i}\left(X_t, t\right) d t+\left(G\left(X_{t}, t\right) d W_{t}\right)\right)+\frac{1}{2} \sum_{i, j}\left(\frac{\partial^{2} \phi}{\partial x_{i} \partial x_{j}}\right)\left[G(X_t, t) Q G^{\top}(X_t, t)\right]_{i j} d t .
            \end{aligned}
    \end{equation}
    Take expectation of both sides:
    \begin{equation}\label{expectation}
        \frac{d E[\phi]}{d t}=\sum_{i} E\left[\frac{\partial \phi}{\partial x_{i}} f_{i}(X_t, t)\right]+\frac{1}{2} \sum_{i j} E\left[\frac{\partial^{2} \phi}{\partial x_{i} \partial x_{j}}\left[G Q G^{\top}\right]_{i j}\right]
    \end{equation}
    So 
    \begin{equation}\left\{
        \begin{aligned}
            &\frac{d E[\phi]}{d t} =\frac{d}{d t}\left[\int \phi(x) P(X_t=x, t) d x\right]=\int \phi(x) \frac{\partial P(x, t)}{\partial t} dx\\
            &\sum_{i} E\left[\frac{\partial \phi}{\partial x_{i}} f_{i}\right]=\sum_{i} \int\frac{\partial \phi}{\partial x_{i}} f_{i}(X_t=x, t) P d x
            =-\sum_{i} \int \phi \cdot \frac{\partial}{\partial x_{i}}\left[f_{i}(x, t) p(x, t)\right] d x . \\
            &\frac{1}{2} \sum_{i j} E\left[\frac{\partial^{2} \phi}{\partial x_{i} \partial x_{j}}\left[G Q G^{\top}\right]_{i j}\right]=\frac{1}{2} \sum_{i j} \int \frac{\partial^{2} \phi}{\partial x_{i} \partial x_{j}}\left[G Q G^{\top}\right]_{i j} P d x
            =\frac{1}{2} \sum_{i j} \int \phi(x) \frac{\partial^{2}}{\partial x_{i} \partial x_{j}}\left(\left[G Q G^{\top}\right]_{i j} P\right) d x. \\
        \end{aligned}\right.
    \end{equation}
    then
    $$\int \phi  \frac{\partial P}{\partial t} d X=-\sum_{i} \int \phi  \frac{\partial}{\partial x_{i}}\left(f_{i} P\right) d X+\frac{1}{2} \sum_{i j} \int \phi \frac{\partial^{2}}{\partial x_{i} x_{j}}\left(\left[G Q G^{\top}\right]_{i j} P\right) d x$$
    Hence $$\int \phi \cdot\left[\frac{\partial P}{\partial t}+\sum_{i} \frac{\partial}{\partial x_{i}}\left(f_{i} P\right)-\frac{1}{2} \sum_{i j} \frac{\partial^{2}}{\partial x_{i} \partial x_{j}}\left(\left[G Q G^{\top}\right]_{i j} P\right)\right] d X=0$$
    Therefore P s.t.    
    \begin{equation}
        \frac{\partial P}{\partial t}+\sum_{i} \frac{\partial}{\partial x_{i}}\left(f_{i}(x, t) P(x, t)\right)-\frac{1}{2} \sum_{i=1} \frac{\partial^{2}}{\partial X_{i} \partial X_{j}}\left(\left[G Q G^{\top}\right]_{i j} P\left(x,t\right)\right)=0
    \end{equation}
    Which gives the FPK Equation.
\end{proof}

\begin{remark}
    When SDE is time independent:  
    \begin{equation}
        d X_t=f(X_t) d t+G(X_t) d W_{t}  
    \end{equation}
    then the solution of FPK often converges to a stationary solution s.t.  $\frac{\partial P}{\partial t}=0$.
\end{remark}
Here is an another way to show FPK equation: Since we have inner product $\langle\phi, \psi\rangle=\int \phi(x)\psi(x)dx$. Then $E[\phi(x)]=\langle\phi, P\rangle$.

As the equation \ref{expectation} can be written as 
\begin{equation}
    \frac{d}{dt}\langle\phi, P\rangle=\langle\mathcal{A}\phi, P\rangle
\end{equation}
Where $\mathcal{A}$ has been mentioned above. If we note the adjoint operator of $\mathcal{A}$ as $\mathcal{A}^*$, then we have
\begin{equation}
    \langle\phi, \frac{dP}{dt}-\mathcal{A}^*(P)\rangle=0,\forall \phi(x)
\end{equation}
Hence we have 
\begin{theorem}[FPK Equation]
    \begin{equation}
    \frac{dP}{dt}=\mathcal{A}^*(P),\operatorname{where} \mathcal{A}^*(\cdot)=-\sum_{i} \frac{\partial}{\partial x_{i}}\left(f_{i}(x, t) (\cdot)\right)+\frac{1}{2} \sum_{i=1} \frac{\partial^{2}}{\partial x_{i} \partial x_{j}}\left(\left[G Q G^{\top}\right]_{i j}(\cdot)\right)
\end{equation}
It can be rewritten as:
\begin{equation}
    \begin{aligned}
        \frac{\partial P}{\partial t} &= -\nabla\cdot\left[f(x, t) p(x, t)\right]+\frac{1}{2} \nabla^2\cdot\left[\left(G Q G^{\top}\right) p(x, t)\right] \\
        &=-\nabla\cdot\left[f(x, t) p(x, t)-\frac{1}{2} \nabla\cdot\left[\left(G Q G^{\top}\right) p(x, t)\right]\right]
    \end{aligned}
\end{equation}
\end{theorem}

\begin{theorem}[Transition Density(Forward Komogorov Equation)]
     The transition density $P_{t|s}(x_t|x_s),t\geq s$, which means the propability of transition from $X(s)=x_s$ to $X(t)=x_t$, satisfies the FPK equation with initial condition $P_{s|s}(x|x_s)=\delta(x-x_s)$
     i.e. for $P_{t|s}(x|y)$, it solves
     \begin{equation}
        \frac{\partial P_{t|s}(x|y)}{\partial t}=\mathcal{A}^*(P_{t|s}(x|y)), \operatorname{with} P_{s|s}(x|y)=\delta(x-y)
     \end{equation}
\end{theorem}

\begin{theorem}[Backward Komogorov Equation]
    $P_{s|t}(y|x)$ for $t\geq s$ solves:
    \begin{equation}
        \frac{\partial P_{s|t}(y|x)}{\partial s} + \mathcal{A}(P_{s|t}(y|x))=0, \operatorname{ with }P_{s|t}(y|x) = \delta(x-y)
    \end{equation}
\end{theorem}

\section{Means and Covariances of SDE}
After we derived the FPK equation, which is the complete probabilistic description of SDE, we can derive the mean and covariance of SDE.
By takeing $\phi(x, t)$, then
\begin{equation}
    \frac{d E[\phi]}{d t}=E\left[\frac{\partial \phi}{\partial t}\right]+\sum_{i} E\left[\frac{\partial \phi}{\partial x_{i}} f_{i}(X_t, t)\right]+\frac{1}{2} \sum_{i j} E\left[\frac{\partial^{2} \phi}{\partial x_{i} \partial x_{j}}\left[G Q G^{\top}\right]_{i j}\right]
\end{equation}
By taking $\phi(X, t)=x_i$ and $\phi(X, t)=x_ix_j-m(t)_im(t)_j$, we have the mean function $m(t)=E[X_t]$ 
and covariance function $c(t)=E\left[\left(X_t-m(t)\right)\left(X_t-m(t)\right)^T\right]$ respectively, s.t.
\begin{equation}\label{SDEMC}
    \left\{
        \begin{aligned}
            &\frac{d m}{d t}=E\left[f(X_t, t)\right]\\
            &\frac{d c}{d t}=E\left[f(X, t)(X-m(t)^T)\right]+E\left[(X-m(t)f^T(X, t))\right]+E\left[G(X_t, t)QG^T(X_t, t)\right]
        \end{aligned}
    \right.
\end{equation}
So we can estimate the mean and covariance of solution to SDE. However, these equations cannot be used as such, 
because only in the Gaussian case do the expectation and covariance actually characterize the distribution. 

\section{Linear SDE}
The linear SDE has explicit solution. Assume the linear SDe has the form 
\begin{equation}
    dX_t =\left(K(t)X_t + B(t)\right)dt + G(t)dW_t
\end{equation}
where $K(t)\in \mathbf{R}^{d\times d}, B(t)\in \mathbf{R}^{d}, G(t)\in \mathbf{R}^{d\times m}$ are given functions. 
$X_t \in \mathbf{R}^d$ is the state vector, $W_t \in \mathbf{R}^m$ is the Brownian Motion with diffusion matrix $Q$.

\begin{theorem}
    The explicit solution to the linear SDE is given by:
    \begin{equation}\label{explicitsolLSDE}
        X_t = \Psi(t, t_0)X_0 + \int_{t_0}^t \Psi(t, s)B(s)ds + \int_{t_0}^t \Psi(t, s)G(s)dW_s
    \end{equation}
    where $\Psi(t, t_0)$ is the transition matrix of the linear SDE, which satisfies the following matrix ODE:
    \begin{equation}
        \frac{d\Psi}{dt} = K(t)\Psi(t, t_0), \Psi(t_0, t_0) = I
    \end{equation}
    Hence, $X_t$ is a Gaussian process(A linear transformation of Brownian Motion which is a Gaussian process).
\end{theorem}
\begin{proof}
    Multiply both sides of the SDE by Integrating factor $\Psi(t_0, t)$ and apply Ito's formula to $\Psi(t_0, t)X_t$.

    See Sarkka P49.
\end{proof}
As discussed above, we can compute the mean and covariance function of solution to linear SDE. 
\begin{theorem}
    The mean and covariance function of solution to linear SDE are given by:
    \begin{equation}
        \left\{
            \begin{aligned}
                &\frac{d m}{d t} = K(t)m(t) + B(t)\\
                &\frac{d c}{d t} = K(t)c(t) + c(t)K^T(t)+ G(t)QG^T(t)
            \end{aligned}
        \right.
    \end{equation}
    with initial condition $m_0 =m(t_0)=E[X_0], c_0 =c(t_0)=Cov(X_0)$. Then the solution is given by solving the above ODEs:
    \begin{equation}\label{LSDEMC}
        \left\{
            \begin{aligned}
                &m(t) = \Psi(t, t_0)m_0 + \int_{t_0}^t \Psi(t, s)B(s)ds\\
                &c(t) = \Psi(t, t_0)c_0\Psi^T(t, t_0) + \int_{t_0}^t \Psi(t, s)G(s)QG^T(s)\Psi^T(t, s)ds
            \end{aligned}
        \right.
    \end{equation}

\end{theorem}
\begin{proof}
    Apply $F(X, t)=K(t)X + B(t), G(X, t)=G(t)$ to \ref{SDEMC}.
\end{proof}

Hence the solution to linear SDE is a Gaussian process with mean and covariance function given by the above ODEs.
\begin{theorem}
The solution to LSDE is Gaussian:
\begin{equation}
    p(X, t) = \mathcal{N}(X(t)|m(t), c(t))
\end{equation}
Specially when $X_0 = x_0$ is fixed, then 
\begin{equation}\label{transitiondensity}
    p(X,t|X_0=x_0) = \mathcal{N}(X(t)|m(t|x_0), c(t|x_0))
\end{equation}
That is, $m_0 = x_0, c_0 = 0$. Then we have:
\begin{equation}
    \left\{
        \begin{aligned}
            &m(t|x_0) = \Psi(t, t_0)x_0 + \int_{t_0}^t \Psi(t, s)B(s)ds\\
            &c(t|x_0) = \int_{t_0}^t \Psi(t, s)G(s)QG^T(s)\Psi^T(t, s)ds
        \end{aligned}
    \right.
\end{equation}
\end{theorem}
\begin{proof}
    The proof is straight foward either by applying $m_0 = x_0, c_0 = 0$ to \ref{LSDEMC} or by eq \ref{explicitsolLSDE}.
\end{proof}
So, to sum up, linear SDE has great properties! The distribution is completedly decided by the inital condition.
Also, if we generate $X_0$ to $X_{t_k}$, which means that we begin SDE at $t_i$ with $X_{t_i}$, we have the equivalent discretization of SDE:
\begin{theorem}
    Original SDE is weakly, in distribution, equivalent to the following discrete-time SDE:
    \begin{equation}\label{DTSDE}
        X_{t_{i+1}} = A_iX_{t_i} + B_i + G_i
    \end{equation} 
    where
    \begin{equation}\left\{
        \begin{aligned}
            A_i &= \Psi(t_{i+1}, t_i)\\
            B_i &= \int_{t_i}^{t_{i+1}} \Psi(t_{i+1}, s)B(s)ds\\
            G_i &= \int_{t_i}^{t_{i+1}} \Psi(t_{i+1}, s)G(s)QG^T(s)\Psi^T(t_{i+1}, s)ds
        \end{aligned}\right.
    \end{equation}
\end{theorem}
\begin{proof}
    The proof is straight forward.
\end{proof}
\begin{theorem}
    The covariance of $X_t$ and $X_s(s<t)$ is given by:
    \begin{equation}
        Cov(X_t, X_s) = \Psi(t, s)c(s)
    \end{equation}
\end{theorem}

\begin{proof}
    See Sarkka P88-89.
\end{proof}

\section{Feynman-Kac Formula}
The Feynman-Kac Formula bridges PDE and certain stochastic value of SDE solutions.

Consider $u(x, t)$ satisfied the following PDE:
\begin{equation}
    \frac{\partial u}{\partial t}+f(x) \frac{\partial u}{\partial x}+\frac{1}{2} L^{2}(x) \frac{\partial^{2} u}{\partial x^{2}}=0 . \quad u(x, T)=\psi(x) .
\end{equation}

Then we define a stochastic process $X(t)$  on  $\left[t^{\prime}, T\right]$  as

\begin{equation}
    d X=f(X) d t+L(X) d W_{t} \quad X\left(t^{\prime}\right)=x^{\prime}
\end{equation}
By Ito formula:
\begin{equation}
    \begin{aligned}
    d u & =\frac{\partial u}{\partial t} d t+\frac{\partial u}{\partial x} d x+\frac{1}{2} \frac{\partial^{2} u}{\partial x^{2}} d x^{2} \\
    & =\frac{\partial u}{\partial t} d t+\frac{\partial u}{\partial x}\left(f(x) d t+L(x) d W_{t}\right)+\frac{1}{2} \frac{\partial^{2} u}{\partial x^{2}} L^{2}(x) d t \\
    & =\left(\frac{\partial u}{\partial t}+\frac{\partial u}{\partial x} f(x)+\frac{1}{2} \frac{\partial^{2} u}{\partial x^{2}} L^{2}(x)\right) d t+\frac{\partial u}{\partial x} L(x) d W_{t} . \\
    & =\frac{\partial u}{\partial x} L(x) d W_{t} .
    \end{aligned}
\end{equation}

Integrating both sises from $t^{\prime}$  to  T:
\begin{equation}
    \begin{aligned}
        \int_{t^{\prime}}^{T} \frac{\partial u}{\partial x} L(x) d W_{t} &= u(X(T), T) - u(X(t'), t')\\
        &= \psi(X(T)) - u(x', t')
    \end{aligned}
\end{equation}
Take expectation of both sides:
\begin{equation}
    u\left(x^{\prime}, t^{\prime}\right)=E[\psi(X(T))]
\end{equation}  
This can be generalized to PDE like:
\begin{equation}
    \frac{\partial u}{\partial t}+f(x) \frac{\partial u}{\partial x}+\frac{1}{2} L^{2}(x) \frac{\partial^{2} u}{\partial x^{2}}-r u=0 . \quad u(x, T)=\psi(x) \text {. }
\end{equation}
By consider the Ito formula of $e^{-rt}u(x, t)$, we can similarly compute the resulting Feynman-Kac equation as 
\begin{equation}
    u(x', t') = e^{-r(T-t')}E\left[\psi(X(T))\right]
\end{equation}
This means we can get the value of PDE at $(x', t')$ by simulating SDE paths beginning at $(x', t')$, and compute corresponding $E\left[\psi(X(T))\right]$. We can get more generalized conclusion:
\begin{algorithm}[Solve Backward PDE]

    To compute the backward PDE: $(\mathcal{A}_t-r)(u)=0$, i.e.
    \begin{equation}
        \frac{\partial u}{\partial t} + \sum_{i} \frac{\partial u(x, t)}{\partial x_{i}} f_{i}(x, t)+\frac{1}{2} \sum_{i, j}\left(\frac{\partial^{2}u(x, t)}{\partial x_{i} \partial x_{j}}\right)\left[G(x, t)Q G^{\top}(x, t)\right]_{i j} - ru(x, t)=0
    \end{equation}
    with boundary condition $u(x, T)=\psi(x)$. Then for any fixed points $(x', t')$ where $t'\leq T, x'\in D$, $u(x', t')$ can be computed as:\\
    Step1. Simulate N sample paths of SDE from $t'$ to $T$:
    \begin{equation}
        dX_t=f(X_t, t)dt + G(X_t, t)dW_t\operatorname{with}X(t')=x'
    \end{equation}
    Step2. Estimate $u(x', t') = e^{-r(T-t')}E\left[\psi(X(T))\right]$
\end{algorithm}

\begin{algorithm}[Solve Forward PDE]
    Consider the solution $u(x, t)$ of forward PDE: $\frac{\partial u}{\partial t}=(\mathcal{A}-r)(u)$, i.e.
    \begin{equation}
        \frac{\partial u}{\partial t}=\sum_{i} \frac{\partial u(x, t)}{\partial x_{i}} f_{i}(x, t)+\frac{1}{2} \sum_{i, j}\left(\frac{\partial^{2}u(x, t)}{\partial x_{i} \partial x_{j}}\right)\left[G(x, t) Q G^{\top}(x, t)\right]_{i j} - ru(x, t)
    \end{equation}
    with initial condition $u(x, 0)=\psi(x)$. Then for any fixed points $(x', t')$ where $t'\leq T, x'\in D$, $u(x', t')$ can be computed as:\\
    Step1. Simulate N sample paths of SDE from $0$ to $t'$:
    \begin{equation}
        dX_t=f(X_t, t)dt + G(X_t, t)dW_t\operatorname{with}X(0)=x'
    \end{equation}
    Step2. Estimate $u(x', t') = e^{-rt'}E\left[\psi(X(t'))\right]$
\end{algorithm}
\begin{algorithm}[Solve Boundary Value Problem]
    For solution $u(x)$ to the following elliptic PDE defined on some domain $D$:
    \begin{equation}
        \sum_{i} \frac{\partial u(x)}{\partial x_{i}} f_{i}(x)+\frac{1}{2} \sum_{i, j}\left(\frac{\partial^{2}u(x)}{\partial x_{i} \partial x_{j}}\right)\left[G(x) Q G^{\top}(x)\right]_{i j} - ru(x)=0
    \end{equation}
    with boundary condition $u(x)=\psi(x)$ on $\partial D$. Then for any fixed points in $D$ can be computed as:\\
    Step1. Simulate N sample paths of SDE from $t'$ to the first exit time $T_e$:
    \begin{equation}
        dX_t=f(X_t)dt + G(X_t)dW_t\operatorname{with}X(t')=x'
    \end{equation}
    Step2. Estimate $u(x') = e^{-r(T_e-t')}E\left[\psi(X(T_e))\right]$
\end{algorithm}

\section{Linear Filtering Problem}

\section{Parameter Estimation in SDE}
Consider the SDE with unknown parameters $\theta$:
\begin{equation}
    dX_t = f(X_t, t, \theta)dt + G(X_t, t, \theta)dW_t
\end{equation}
The diffusion matrix of $W_t$ might also depend on $\theta$, i.e. $Q=Q(\theta)$.
The goal is to estimate the unknown parameter $\theta$ by observing the SDE solution $X_t$.
\begin{definition}
    Maximum likelihood method:
    Give the observattion $X_{t_0}, X_{t_1}, \cdots, X_{t_n}$, the likelihood function is given by:
    \begin{equation}
        p(X_{t_0}, X_{t_1}, \cdots, X_{t_n}|\theta) = \prod_{i=0}^{n} p(X_{t_i}|X_{t_{i-1}}, \theta)
    \end{equation}
    where $p(X_{t_i}|X_{t_{i-1}}, \theta)$ is the transition density of SDE.

    Then to maximize the likelihood function, it is equal to minimize the negative log-likelihood function:
    \begin{equation}
        L(\theta) = -\log p(\theta) = -\sum_{i=0}^{n} \log p(X_{t_i}|X_{t_{i-1}}, \theta)
    \end{equation}
    Then the posterior distribution of $\theta$ is given by:
    \begin{equation}
        \begin{aligned}
            p(\theta|X_{t_0}, X_{t_1}, \cdots, X_{t_n}) &= \frac{p(X_{t_0}, X_{t_1}, \cdots, X_{t_n}|\theta)p(\theta)}{p(X_{t_0}, X_{t_1}, \cdots, X_{t_n})}\\
            &= \frac{p(X_{t_0}, X_{t_1}, \cdots, X_{t_n}|\theta)p(\theta)}{p(X_{t_0}, X_{t_1}, \cdots, X_{t_n})}\\
            &\propto \prod_{i=0}^{n} p(X_{t_i}|X_{t_{i-1}}, \theta)p(\theta)
        \end{aligned}
    \end{equation}
    where $p(\theta)$ is the prior distribution of $\theta$. So we have \textbf{ML} and \textbf{MAP} estimation.
\end{definition}

We find that the difficulty of ML and MAP estimation is to compute the transition density $p(X_{t_i}|X_{t_{i-1}}, \theta)$, which is given by FPK equation. 
However, we know that FPK equation is too difficult to solve. So we need to use some approximation methods to estimate the transition density.
Among this we find that the transition density of linear SDE can be explicitly computed! That is great!
\begin{theorem}
    The transition density of linear SDE is given by eq \ref{transitiondensity}. Then the negative log-likelihood is given by:
    \begin{equation}
        L(\theta) = \sum_{i=0}^{T - 1} \left[\frac{1}{2}\log(|2\pi G_i|)+\frac{1}{2}(X_{t_{i+1}} - A_i(\theta)X_{t_i} - B_i(\theta))^\top G_i^{-1}(X_{t_{i+1}} - A_i(\theta)X_{t_i} - B_i(\theta))\right]
    \end{equation}
    where $A_i, B_i, G_i$ are given in eq \ref{DTSDE}.
\end{theorem}

\section{Conservation Laws}
\begin{theorem}Two important theorems in calculus:

    1.\textbf{Divergence Theorem}:
    \begin{equation}
        \int_{\Omega} \nabla \cdot \mathbf{F} d x = \int_{\partial \Omega} \mathbf{F} \cdot \mathbf{n} d S
    \end{equation}

    2.\textbf{Reynolds Transport Theorem}:
    \begin{equation}
        \frac{d}{dt}\int_{\Omega(t)} f(t, x) d x = \int_{\Omega (t)} \frac{\partial f}{\partial t} d x + \int_{\partial \Omega (t)} f(t, x) \mathbf{v} \cdot \mathbf{n} d S
    \end{equation}
    where $u$ is the velocity at $\partial \Omega (t)$.
\end{theorem}
Here the $\Omega(t)$ is the domain of the flow, and the $\partial \Omega(t)$ is the boundary of the flow, which is described by the flow map $\phi_s^t$. Here is the definition.
\begin{definition}[Flow Map]
    Assume a description of some characteristic of particle $\mathbf{P}$, like the position or the boundary, as $\mathbf{x}\in \mathcal{R}^m$, then we have a flow map $\phi_s^t(\mathbf{x})\in \mathcal{R}^m$, 
    which means that the flow transimits the characteristic(position) $\mathbf{x}$ from $\mathbf{x}$ at $s$ to $\phi_s^t(\mathbf{x})$ at $t$, controlled by the vector field(velocity field) $\mathbf{F}: \mathcal{R}^m\times \mathcal{R}\to \mathcal{R}^m$:
    \begin{equation}\left\{
        \begin{aligned}
            \frac{d\phi_s^t(\mathbf{x})}{dt} &= \mathbf{F}(\phi_s^t(\mathbf{x}), t)\\
            \phi_s^s(\mathbf{x}) &= \mathbf{x}
        \end{aligned}\right.
    \end{equation}
\end{definition}

If we assume $\Omega(t)$ is composed of particles, i.e. $\Omega(t)=\phi_{t_0}^t(\Omega)$(when $t = t_0$, $\Omega(t_0)=\Omega$), 
then we by \textbf{conservation of mass}, we have the following theorem:
\begin{theorem}[Continuity Equation]
    By conservation of mass, i.e. $\int_{\Omega(t)} \rho(t, \mathbf{x}) d\mathbf{x} = C$, we have:
    \begin{equation}
        \begin{aligned}
            \frac{d}{dt}\int_{\Omega(t)} \rho(t, \mathbf{x}) d\mathbf{x} &= \int_{\Omega(t)} \frac{\partial \rho}{\partial t} d\mathbf{x} + \int_{\partial \Omega(t)} \rho(t, \mathbf{x}) \mathbf{u} \cdot \mathbf{n} d S\\ 
            &= \int_{\Omega(t)} \left(\frac{\partial \rho}{\partial t}  + \nabla\cdot \left(\rho\mathbf{u}\right)\right) d\mathbf{x}=0
        \end{aligned}
    \end{equation}
    Therefore:
    \begin{equation}
        \frac{\partial \rho}{\partial t}  + \nabla\cdot \left(\rho\mathbf{u}\right) = 0
    \end{equation}
    which is also called \textbf{continuity equation}.
\end{theorem}
\begin{theorem}[Conservation of Momentum]
    By conservation of momentum, i.e. 
    \begin{equation}
        \frac{d}{dt}\int_{\Omega(t)} \rho(t, \mathbf{x})\mathbf{v}(t, \mathbf{x}) d\mathbf{x} = -\int_{\partial \Omega(t)} p\cdot \mathbf{n} d S
    \end{equation}
    we have:
    \begin{equation}
        \frac{\partial (\rho \mathbf{v})}{\partial t} + \nabla\cdot \left(\rho \mathbf{v} \otimes \mathbf{v} + p\right) = 0
    \end{equation}
    where $p$ is the pressure.
\end{theorem}
\begin{theorem}[Conservation of Energy]
    \begin{equation}
        \frac{\partial E}{\partial t} + \nabla\cdot \left(\mathbf{v}(E + p)\right) = 0
    \end{equation}
\end{theorem}
Then we have can get the Euler's equation:
\begin{theorem}[Euler's Equation]
    The Euler's equation is given by:
    \begin{equation}
        \frac{\partial}{\partial t}\begin{bmatrix}
        \rho\\ \rho \mathbf{v}\\E
    \end{bmatrix}
    + \nabla\cdot \begin{bmatrix}
        \rho \mathbf{v}\\ \rho \mathbf{v}\otimes \mathbf{v} + p\\ \mathbf{v}(E + p)
    \end{bmatrix} = 0
\end{equation}
So the general form of conservation laws is given by: suppose $U\in \mathcal{R}^d$ is the conserved quantity, $F$ is $\mathcal{R}^d\to \mathcal{R}^d$ is the flux, 
then we have:
\begin{equation}
    \frac{\partial U}{\partial t} + \nabla\cdot \left(F(U)\right) = 0
\end{equation}

\end{theorem}



\end{document} 