\chapter{SDE}
\section{Introduction to SDEs}
\subsection{SODEs}
\begin{problem}
    Assume we have a Stochastic Differential Equation like:
    \begin{equation}\label{sde}
        dX_t = f(X_t, t)dt + G(X_t, t)dW_t
    \end{equation}
    where $X_t\in \mathbf{R}^d,f\in \mathcal{L}(\mathbf{R}^{d+1}, \mathbf{R}^d)$, and $W_t$ is m-dim Brownian Motion with diffusion matrix $Q$, 
    $G(X_t, t)\in \mathcal{L}(\mathbf{R}^{m+1}, \mathbf{R}^d)$, with initial condition $X_0\sim p(X_0)$.
\end{problem}
\subsection{The It'so and Stratonovich Stochastic Integrals}
\subsection{Ito's Formula}
\begin{theorem}[Ito's Formula]
    Let $X_t$ be an Ito process defined by \ref{sde}, and let $\phi(X_t)$ be a twice continuously differentiable function of $x$ and $t$. Then the process $f(X_t, t)$ is also an Ito process, and its dynamics are given by:
    \begin{equation}
        d\varphi(X_t) = \nabla \varphi(X_t)\cdot dX_t + \frac{1}{2}\operatorname{Tr}\left(GQG^T\nabla^2 \varphi(X_t)dt\right)
    \end{equation}
\end{theorem}


\subsection{Mean and Covariance}
We can derive the mean and covariance of SDE.
By applying Ito's formula to $\phi(x, t)$, then
\begin{equation}
    \frac{d E[\phi]}{d t}=E\left[\frac{\partial \phi}{\partial t}\right]+\sum_{i} E\left[\frac{\partial \phi}{\partial x_{i}} f_{i}(X_t, t)\right]+\frac{1}{2} \sum_{i j} E\left[\frac{\partial^{2} \phi}{\partial x_{i} \partial x_{j}}\left[G Q G^{\top}\right]_{i j}\right]
\end{equation}
By taking $\phi(X, t)=x_i$ and $\phi(X, t)=x_ix_j-m(t)_im(t)_j$, we have the mean function $m(t)=E[X_t]$ 
and covariance function $c(t)=E\left[\left(X_t-m(t)\right)\left(X_t-m(t)\right)^T\right]$ respectively, s.t.
\begin{equation}\label{SDEMC}
    \left\{
        \begin{aligned}
            &\frac{d m}{d t}=E\left[f(X_t, t)\right]\\
            &\frac{d c}{d t}=E\left[f(X, t)(X-m(t)^T)\right]+E\left[(X-m(t)f^T(X, t))\right]+E\left[G(X_t, t)QG^T(X_t, t)\right]
        \end{aligned}
    \right.
\end{equation}
So we can estimate the mean and covariance of solution to SDE. However, these equations cannot be used as such, 
because only in the Gaussian case do the expectation and covariance actually characterize the distribution. 

The linear SDE has explicit solution. Assume the linear SDe has the form 
\begin{equation}
    dX_t =\left(K(t)X_t + B(t)\right)dt + G(t)dW_t
\end{equation}
where $K(t)\in \mathbf{R}^{d\times d}, B(t)\in \mathbf{R}^{d}, G(t)\in \mathbf{R}^{d\times m}$ are given functions. 
$X_t \in \mathbf{R}^d$ is the state vector, $W_t \in \mathbf{R}^m$ is the Brownian Motion with diffusion matrix $Q$.

\begin{theorem}
    The explicit solution to the linear SDE is given by:
    \begin{equation}\label{explicitsolLSDE}
        X_t = \Psi(t, t_0)X_0 + \int_{t_0}^t \Psi(t, s)B(s)ds + \int_{t_0}^t \Psi(t, s)G(s)dW_s
    \end{equation}
    where $\Psi(t, t_0)$ is the transition matrix of the linear SDE, which satisfies the following matrix ODE:
    \begin{equation}
        \frac{d\Psi}{dt} = K(t)\Psi(t, t_0), \Psi(t_0, t_0) = I
    \end{equation}
    Hence, $X_t$ is a Gaussian process(A linear transformation of Brownian Motion which is a Gaussian process).
\end{theorem}
\begin{proof}
    Multiply both sides of the SDE by Integrating factor $\Psi(t_0, t)$ and apply Ito's formula to $\Psi(t_0, t)X_t$.

    See Sarkka P49.
\end{proof}
As discussed above, we can compute the mean and covariance function of solution to linear SDE. 
\begin{theorem}
    The mean and covariance function of solution to linear SDE are given by:
    \begin{equation}
        \left\{
            \begin{aligned}
                &\frac{d m}{d t} = K(t)m(t) + B(t)\\
                &\frac{d c}{d t} = K(t)c(t) + c(t)K^T(t)+ G(t)QG^T(t)
            \end{aligned}
        \right.
    \end{equation}
    with initial condition $m_0 =m(t_0)=E[X_0], c_0 =c(t_0)=Cov(X_0)$. Then the solution is given by solving the above ODEs:
    \begin{equation}\label{LSDEMC}
        \left\{
            \begin{aligned}
                &m(t) = \Psi(t, t_0)m_0 + \int_{t_0}^t \Psi(t, s)B(s)ds\\
                &c(t) = \Psi(t, t_0)c_0\Psi^T(t, t_0) + \int_{t_0}^t \Psi(t, s)G(s)QG^T(s)\Psi^T(t, s)ds
            \end{aligned}
        \right.
    \end{equation}

\end{theorem}
\begin{proof}
    Apply $F(X, t)=K(t)X + B(t), G(X, t)=G(t)$ to \ref{SDEMC}.
\end{proof}

Hence the solution to linear SDE is a Gaussian process with mean and covariance function given by the above ODEs.
\begin{theorem}
The solution to LSDE is Gaussian:
\begin{equation}
    p(X, t) = \mathcal{N}(X(t)|m(t), c(t))
\end{equation}
Specially when $X_0 = x_0$ is fixed, then 
\begin{equation}\label{transitiondensity}
    p(X,t|X_0=x_0) = \mathcal{N}(X(t)|m(t|x_0), c(t|x_0))
\end{equation}
That is, $m_0 = x_0, c_0 = 0$. Then we have:
\begin{equation}
    \left\{
        \begin{aligned}
            &m(t|x_0) = \Psi(t, t_0)x_0 + \int_{t_0}^t \Psi(t, s)B(s)ds\\
            &c(t|x_0) = \int_{t_0}^t \Psi(t, s)G(s)QG^T(s)\Psi^T(t, s)ds
        \end{aligned}
    \right.
\end{equation}
\end{theorem}
\begin{proof}
    The proof is straight foward either by applying $m_0 = x_0, c_0 = 0$ to \ref{LSDEMC} or by eq \ref{explicitsolLSDE}.
\end{proof}
So, to sum up, linear SDE has great properties! The distribution is completedly decided by the inital condition.
Also, if we generate $X_0$ to $X_{t_k}$, which means that we begin SDE at $t_i$ with $X_{t_i}$, we have the equivalent discretization of SDE:
\begin{theorem}
    Original SDE is weakly, in distribution, equivalent to the following discrete-time SDE:
    \begin{equation}\label{DTSDE}
        X_{t_{i+1}} = A_iX_{t_i} + B_i + G_i
    \end{equation} 
    where
    \begin{equation}\left\{
        \begin{aligned}
            A_i &= \Psi(t_{i+1}, t_i)\\
            B_i &= \int_{t_i}^{t_{i+1}} \Psi(t_{i+1}, s)B(s)ds\\
            G_i &= \int_{t_i}^{t_{i+1}} \Psi(t_{i+1}, s)G(s)QG^T(s)\Psi^T(t_{i+1}, s)ds
        \end{aligned}\right.
    \end{equation}
\end{theorem}
\begin{proof}
    The proof is straight forward.
\end{proof}
\begin{theorem}
    The covariance of $X_t$ and $X_s(s<t)$ is given by:
    \begin{equation}
        Cov(X_t, X_s) = \Psi(t, s)c(s)
    \end{equation}
\end{theorem}

\begin{proof}
    See Sarkka P88-89.
\end{proof}

\section{Fokker-Planck-Kolmogorov Equation}
\subsection{FPK Equation}
\begin{definition}[Generator]
    The infinitesimal generator of a stochastic process $X(t)$ for function $\phi(x)$, i.e. $\phi(X_t)$ can be defined as
    \begin{equation}
        \mathcal{A} \phi(X_t)=\lim _{s \rightarrow 0^{+}} \frac{E[\phi(X(t+s)]-\phi(X(t))}{s}
    \end{equation}
    Where  $\phi$  is a suitable regular function.
\end{definition}
This leads to Dynkin's Formula very naturally.
\begin{theorem}[Dynkin's Formula]
    \begin{equation}
        E[f(X_t)]=f(X_0)+E\left[\int_0^t\mathcal{A}(f(X_s))ds\right]
    \end{equation}
\end{theorem}

\begin{theorem}
    If  $X(t)$  s.t. \ref{sde}, then the generator is given:
\begin{equation}
    \mathcal{A}(\cdot)=\sum_{i} \frac{\partial(\cdot)}{\partial x_{i}} f_{i}(X_t, t)+\frac{1}{2} \sum_{i, j}\left(\frac{\partial^{2}(\cdot)}{\partial x_{i} \partial x_{j}}\right)\left[G(X_t, t)Q G^{\top}(X_t, t)\right]_{i j}
\end{equation}
\end{theorem}
\begin{proof}
    See P119 of SDE by Oksendal.
\end{proof}

\begin{example}
    If $dX_t=dW_t$, then $\mathcal{A}=\frac{1}{2}\Delta$, where $\Delta$ is the Laplace operator.
\end{example}

\begin{definition}[Generalized Generator]
    For $\phi(x, t)$, i.e. $\phi(X_t, t)$, the generator can be defined as:
    \begin{equation}
        A_{t} \phi(x, t)=\lim _{s \rightarrow 0^{+}} \frac{E[\phi(X(t+s), t+s)]-\phi(X(t), t)}{s}
    \end{equation}
\end{definition}

\begin{theorem}
    Similarly if $X(t)$ s.t. \ref{sde}, then the generalized generator is given:
    \begin{equation}
        \mathcal{A}_t(\cdot)=\frac{\partial(\cdot)}{\partial t}+\sum_{i} \frac{\partial(\cdot)}{\partial x_{i}} f_{i}(X_t, t)+\frac{1}{2} \sum_{i, j}\left(\frac{\partial^{2}(\cdot)}{\partial x_{i} \partial x_{j}}\right)\left[G(X_t, t) Q G^{\top}(X_t, t)\right]_{i j}
    \end{equation}
\end{theorem}
We want to consider the density distribution of $X_t, P(x, t)$
\begin{theorem}[Fokken-Planck-Kolmogorov equation]
    The density function $P(x, t)$ of $X_t$ s.t. \ref{sde} solves the PDE:
    \begin{equation}
        \frac{\partial P(x, t)}{\partial t}=-\sum_{i} \frac{\partial}{\partial x_{i}}\left[f_{i}(x, t) p(x, t)\right]+\frac{1}{2} \sum_{i, j} \frac{\partial^{2}}{\partial x_{i} \partial x_{j}}\left[\left(G Q G^{\top}\right)_{i j} P(x, t)\right]
    \end{equation}
    The PDE is called FPK equation / forwand Kolmogorov equation.
\end{theorem}
\begin{proof}
    Consider the function $\phi(x)$, let $x=X_t$ and apply Ito's Formula:
    \begin{equation}
        \begin{aligned}
            d \phi & =\sum_{i} \frac{\partial \phi}{\partial x_{i}} d x_{i}+\frac{1}{2} \sum_{i, j}\left(\frac{\partial^{2} \phi}{\partial x_{i} \partial x_{j}}\right) d x_{i} d x_{j} \\
            & =\sum_{i} \frac{\partial \phi}{\partial x_{i}}\left(f_{i}\left(X_t, t\right) d t+\left(G\left(X_{t}, t\right) d W_{t}\right)\right)+\frac{1}{2} \sum_{i, j}\left(\frac{\partial^{2} \phi}{\partial x_{i} \partial x_{j}}\right)\left[G(X_t, t) Q G^{\top}(X_t, t)\right]_{i j} d t .
            \end{aligned}
    \end{equation}
    Take expectation of both sides:
    \begin{equation}\label{expectation}
        \frac{d E[\phi]}{d t}=\sum_{i} E\left[\frac{\partial \phi}{\partial x_{i}} f_{i}(X_t, t)\right]+\frac{1}{2} \sum_{i j} E\left[\frac{\partial^{2} \phi}{\partial x_{i} \partial x_{j}}\left[G Q G^{\top}\right]_{i j}\right]
    \end{equation}
    So 
    \begin{equation}\left\{
        \begin{aligned}
            &\frac{d E[\phi]}{d t} =\frac{d}{d t}\left[\int \phi(x) P(X_t=x, t) d x\right]=\int \phi(x) \frac{\partial P(x, t)}{\partial t} dx\\
            &\sum_{i} E\left[\frac{\partial \phi}{\partial x_{i}} f_{i}\right]=\sum_{i} \int\frac{\partial \phi}{\partial x_{i}} f_{i}(X_t=x, t) P d x
            =-\sum_{i} \int \phi \cdot \frac{\partial}{\partial x_{i}}\left[f_{i}(x, t) p(x, t)\right] d x . \\
            &\frac{1}{2} \sum_{i j} E\left[\frac{\partial^{2} \phi}{\partial x_{i} \partial x_{j}}\left[G Q G^{\top}\right]_{i j}\right]=\frac{1}{2} \sum_{i j} \int \frac{\partial^{2} \phi}{\partial x_{i} \partial x_{j}}\left[G Q G^{\top}\right]_{i j} P d x
            =\frac{1}{2} \sum_{i j} \int \phi(x) \frac{\partial^{2}}{\partial x_{i} \partial x_{j}}\left(\left[G Q G^{\top}\right]_{i j} P\right) d x. \\
        \end{aligned}\right.
    \end{equation}
    then
    $$\int \phi  \frac{\partial P}{\partial t} d X=-\sum_{i} \int \phi  \frac{\partial}{\partial x_{i}}\left(f_{i} P\right) d X+\frac{1}{2} \sum_{i j} \int \phi \frac{\partial^{2}}{\partial x_{i} x_{j}}\left(\left[G Q G^{\top}\right]_{i j} P\right) d x$$
    Hence $$\int \phi \cdot\left[\frac{\partial P}{\partial t}+\sum_{i} \frac{\partial}{\partial x_{i}}\left(f_{i} P\right)-\frac{1}{2} \sum_{i j} \frac{\partial^{2}}{\partial x_{i} \partial x_{j}}\left(\left[G Q G^{\top}\right]_{i j} P\right)\right] d X=0$$
    Therefore P s.t.    
    \begin{equation}
        \frac{\partial P}{\partial t}+\sum_{i} \frac{\partial}{\partial x_{i}}\left(f_{i}(x, t) P(x, t)\right)-\frac{1}{2} \sum_{i=1} \frac{\partial^{2}}{\partial x_{i} \partial x_{j}}\left(\left[G Q G^{\top}\right]_{i j} P\left(x,t\right)\right)=0
    \end{equation}
    Which gives the FPK Equation.
\end{proof}

\begin{remark}
    When SDE is time independent:  
    \begin{equation}
        d X_t=f(X_t) d t+G(X_t) d W_{t}  
    \end{equation}
    then the solution of FPK often converges to a stationary solution s.t.  $\frac{\partial P}{\partial t}=0$.
\end{remark}
Here is an another way to show FPK equation: Since we have inner product $\langle\phi, \psi\rangle=\int \phi(x)\psi(x)dx$. Then $E[\phi(x)]=\langle\phi, P\rangle$.

As the equation \ref{expectation} can be written as 
\begin{equation}
    \frac{d}{dt}\langle\phi, P\rangle=\langle\mathcal{A}\phi, P\rangle
\end{equation}
Where $\mathcal{A}$ has been mentioned above. If we note the adjoint operator of $\mathcal{A}$ as $\mathcal{A}^*$, then we have
\begin{equation}
    \langle\phi, \frac{dP}{dt}-\mathcal{A}^*(P)\rangle=0,\forall \phi(x)
\end{equation}
Hence we have 
\begin{theorem}[FPK Equation]
    \begin{equation}
    \frac{dP}{dt}=\mathcal{A}^*(P),\operatorname{where} \mathcal{A}^*(\cdot)=-\sum_{i} \frac{\partial}{\partial x_{i}}\left(f_{i}(x, t) (\cdot)\right)+\frac{1}{2} \sum_{i=1} \frac{\partial^{2}}{\partial x_{i} \partial x_{j}}\left(\left[G Q G^{\top}\right]_{i j}(\cdot)\right)
\end{equation}
It can be rewritten as:
\begin{equation}
    \begin{aligned}
        \frac{\partial P}{\partial t} &= -\nabla\cdot\left[f(x, t) P(x, t)\right]+\frac{1}{2} \nabla^2\cdot\left[\left(G Q G^{\top}\right) P(x, t)\right] \\
        &=-\nabla\cdot\left[f(x, t) P(x, t)-\frac{1}{2} \nabla\cdot\left[\left(G Q G^{\top}\right) P(x, t)\right]\right]
    \end{aligned}
\end{equation}

We define the probability flux to be:
\begin{equation}
    J(x, t) = f(x, t) p(x, t)-\frac{1}{2} \nabla\cdot\left[M(x) p(x, t)\right], M(x)=G(x, t)Q(x, t)G(x, t)^T
\end{equation}
Integrating the Fokker-Planck equation over $\mathbb{R}^d$ and using the divergence theorem on the right hand side of  the equation, we have:
\begin{equation}
    \frac{d}{dt}\int_{R^d}p(x, t)dx=\int_{R^d}\nabla\cdot J(x, t)dx=0
\end{equation}
The stationary Fokker-Planck equation, whose solutions give us the invariant distributions of the diffusion process $X_t$, can be written in the form
\begin{equation}
    \nabla \cdot J(x, t)=0
\end{equation}
Consequently, the equilibrium probability flux is a divergence-free vector field.

\end{theorem}

\subsection{Forward and backward Komogorov Equation}
\begin{theorem}
    Fix $t>s$, let $u(x, s):= E\left[g(X_t)|X_s=x\right]=\int g(y)P(y, t|x, s)dy$, then $u(x, s)$ satisfies the following equation:
    \begin{equation}
        \frac{\partial u}{\partial s}+f(x, s)\cdot \nabla u+\frac{1}{2}\nabla \cdot (M\nabla u)=0, \qquad u(x, s)=g(x)
    \end{equation}
\end{theorem}

\begin{theorem}[Transition Density(Forward Komogorov Equation)]
     The transition density $P_{t|s}(x_t|x_s),t\geq s$, which means the propability of transition from $X(s)=x_s$ to $X(t)=x_t$, satisfies the FPK equation with initial condition $P_{s|s}(x|x_s)=\delta(x-x_s)$
     i.e. for $P_{t|s}(x|y)$, it solves
     \begin{equation}
        \frac{\partial P_{t|s}(x|y)}{\partial t}=\mathcal{A}^*(P_{t|s}(x|y)), \operatorname{with} P_{s|s}(x|y)=\delta(x-y)
     \end{equation}
\end{theorem}




\subsection{Ornstein-Uhlenbeck Process}
\begin{definition}[Ornstein-Uhlenbeck Process]
    The Ornstein-Uhlenbeck Process is defined as:
    \begin{equation}
        dX_t = -\alpha X_t dt+\sqrt{2D} dW_t
    \end{equation}
    where $\alpha>0, D>0$, normally $D = \frac{1}{\beta}$.
\end{definition}
By FPK equation, we have:
\begin{equation}\left\{
    \begin{aligned}
        &\frac{\partial p}{\partial t} = \alpha \frac{\partial}{\partial x}(xp)+D\frac{\partial^2 p}{\partial x^2}\\
        &p_{0}(x|x_0) = \delta(x-x_0)
    \end{aligned}\right.\label{OU}
\end{equation}
When (\ref{OU}) is  used to model the velocity or position of a particle, the noisy term on the right hand side of the equation is related to thermal fluctuations.
The solution of (\ref{OU}) can be computed:
\begin{equation}
    X_t \sim N(x_0e^{-\alpha t}, \frac{D}{\alpha}(1-e^{-2\alpha t}))
\end{equation}
The generator of OU process is:
\begin{equation}
    \mathcal{L}=-\alpha x\cdot \nabla +D\Delta
\end{equation}

We need to study the properties of the generator $\mathcal{L}$. When the unique invariant density of OU is $\rho$, do transformation:
\begin{equation}
    \mathcal{L}^*(h\rho)=\rho \mathcal{L}h
\end{equation}
The IVP for FPK equation:
\begin{equation}
    \frac{\partial p}{\partial t}=\mathcal{L}^*p,\qquad p(x,0)=p_0(x)
\end{equation}
becomes:
\begin{equation}
    \frac{\partial h}{\partial t}=\mathcal{L}h,\qquad h(x,0)=\rho^{-1}p_0(x)
\end{equation}

\begin{theorem}
    Consider the eigenpairs problem for the generator operator $\mathcal{L}$ of OU process:
    \begin{equation}\left\{
        \begin{aligned}
            &\lambda_n = \alpha n\\
            &\phi_n(x) = \frac{1}{n!}H_n(\sqrt{\alpha \beta}x)
        \end{aligned}\right.\qquad n=0, \cdots, \infty \label{eigenpairs}
    \end{equation}
    where $H_n(x)$ is the $n$-th Hermite polynomial:
    \begin{equation}
        H_n(x) = (-1)^n e^{x^2/2}\frac{d^n}{dx^n}(e^{-x^2/2})
    \end{equation}
\end{theorem}


\subsection{Langevin SDE}
The Langevin SDE has the following form:
\begin{equation}
    X_{t + s} = X_t + \nabla \log p_t(x_t)s + \sqrt{2s}\xi
\end{equation}
where $X_t\in \mathcal{R}^d, p_t(x_t)=p(X_t=x_t)$, $\xi\sim N(0, I)$, $I$ is identical matrix of $m \times m$. Our goal is to sample from specific $p(x, t)$.

\begin{theorem}
    The density of Langevin Diffusion Model converges to $p(x)$ over time. In other words, if $X_t\sim p(x)$, then $X_{t+s}\sim p(x)$ for $\forall s>0$. 
\end{theorem}
\begin{proof}
    Let  $\mu_{t}(f)=E\left[f\left(X_{t}\right)\right]$. Consider  $\mu_{t+\tau}(f)=E\left[f\left(X_{t+\tau}\right)\right]$, as $\tau \rightarrow 0$. Then  
\begin{equation}
    \begin{aligned}
        \mu_{t+\tau}=&E\left[f\left(X_{t}+\nabla \log p_t\left(x_{t}\right) \cdot \tau+\sqrt{2 \tau} \xi\right)\right]\\
        =&E\left[f\left(x_{t}\right)+\nabla^{\top} f\left(x_{t}\right)\left(\tau \nabla \log p_t\left(x_{t}\right)+\sqrt{2 \tau} \xi\right)\right. \\
        &+\frac{1}{2}\left.\left(\nabla^{\top}\log p_t(x_t)\tau + \sqrt{2\tau}\xi\right)\nabla^2f(x_t)\nabla\log p_t(x_t)\tau + \sqrt{2\tau}\xi\right]\\
        =&E\left[f\left(x_{t}\right)\right]+E\left[\tau \nabla^{\top}f\left(x_{t}\right) \nabla \log p_t\left(x_{t}\right)\right]\\
        &+\frac{\tau^{2}}{2} E\left[\nabla^{\top} \log p\left(x_{t}\right) \cdot \nabla^{2} f\left(x_{t}\right) \cdot \nabla \log p\left(x_{t}\right)\right] +E\left[\tau \xi^{\top} \nabla^{2} f\left(x_{t}\right) \xi\right]
    \end{aligned}
\end{equation}
 
The second term:
\begin{equation}
    \begin{aligned}
        &\tau E\left[\nabla^{\top} f \nabla \log p_{t}\right] \\
        =&\tau \int \nabla f \cdot \nabla \log p_{t} p_{t} d x=\tau \int \nabla f \cdot \nabla p_{t} d x \\
        =&-\tau \int \operatorname{tr}\left(\nabla^{2} f\right) \cdot p_{t} d x=-\tau E\left[\operatorname{tr}\left(\nabla^{2} f\right)\right]\\
        =&-\tau E\left[\xi^{\top} \nabla^{2} f \xi\right] \\
    \end{aligned}
\end{equation}
Then 
\begin{equation}
    \mu_{t+\tau} =E\left[\frac{1}{2} \nabla^{\top} \log p_{t} \nabla^{2} f \nabla \log p_{t}\right] \cdot \tau^{2}=O\left(\tau^{2}\right)
\end{equation}
Hence we have $\frac{d}{dt}(\mu_t)=0$, i.e. $E[\mu_t]=E[\mu_{t+s}]$ for $\forall s>0$.
\end{proof}

\begin{remark}
    We define the density of normal distribution $N(x ; \mu, \Sigma)$, and its log-density, gradient of density and score as follows:
    \begin{equation}\left\{
        \begin{aligned}
            &N(x ; \mu, \Sigma)=\frac{1}{\sqrt{(2 \pi)^{d}|\Sigma|}} e^{-\frac{1}{2}(x-\mu)^{\top} \Sigma^{-1}(x-\mu)}\\
            &\log N(x ; \mu, \Sigma)=-\frac{1}{2}(x-\mu)^{\top} \Sigma^{-1}(x-\mu)-\log \left(\sqrt{(2 \pi)^{d}|\Sigma|}\right) . \\
            &\nabla_{x} N(x ; \mu, \Sigma)=N(x ; \mu, \Sigma)\Sigma^{-1}(x-\mu) \\
            &\nabla_{x} \log N(x ; \mu, \Sigma)=-\Sigma^{-1}(x-\mu) .
        \end{aligned}\right.
    \end{equation}
\end{remark}

Actually, Langevin SDE is not necessary be as above i.e. the diffusion term is not necessary to be $\sqrt{2}$. The reason is to guarantee the stationary distribution of $p_t(x)$.
i.e. the term $\frac{\partial p(x,t)}{\partial t}=0$ in FPK equation. If the diffusion term is $g(t)$, then by FPK equation, we have 
$$\nabla_x\cdot(fp-\frac{1}{2}g^2(t)\nabla p)=0$$
then $f(x,t) = \frac{1}{2}g^2(t)\frac{\nabla_x p(x, t)}{p(x,t)}=\frac{1}{2}g^2(t)\nabla_x\log p(x, t)$.

\section{What is Diffusion After All?}
\subsection{From SDEs}
At the beginning, the diffusion phenomenon is observed through the motion of particles(Brownian motion). 
Normally, the SDE can be written as:
\begin{equation}
    dX_t = f(X_t, t)dt + G(X_t, t)dW_t
\end{equation}
Here, we skip the drift term $f(X_t, t)$ and only consider the diffusion term $G(X_t, t)dW_t$, i.e.
\begin{equation}
    dX_t = G(X_t, t)dW_t
\end{equation}
Then by FPK equation, we can derive 
\begin{theorem}
    The probability density function $p(x, t)$ satisfies:
\begin{equation}
    \frac{\partial p(x, t)}{\partial t} = \frac{1}{2} \sum_{i, j} \frac{\partial^{2}}{\partial x_{i} \partial x_{j}}\left[\left(G Q G^{\top}\right)_{i j} p(x, t)\right]=\frac{1}{2}\nabla \cdot \left(\nabla\cdot (GQG^Tp(x, t))\right)
\end{equation}
Specially, when $G(X_t, t)=G(t)$ and $Q=I$, we have:
\begin{equation}
    \frac{\partial p}{\partial t} = \nabla \cdot \left(\frac{GG^T}{2}\nabla p\right)
\end{equation}
\end{theorem}
So, when $X_0\sim p_0$, we can then compute the diffusion density $p(x, t)$ by solving the FPK equation.
\subsection{From Flow Map}
Since we have the definition of \textbf{Flow Map} $\phi_s^t(\mathbf{x})$, which is controlled by vector field $V(\phi_s^t(\mathbf{x}), t)$, 
then just think the $\phi_0^t(\mathbf{x})$ as the trajectory of the particle beginning at $x$ over time, noted as $\phi_t(x)$.
Then the vector field is actually the velocity field of the particle, so we have:
\begin{equation}\left\{
    \begin{aligned}
        \frac{\partial \phi_t(\mathbf{x})}{\partial t} &= V(\phi_t(\mathbf{x}), t)\\
        \phi_0(\mathbf{x}) &= \mathbf{x}
    \end{aligned}\right.
\end{equation}
The motion of particles described by $\phi_t$ determines how the density $p_t(x)$ evolves over time. 
\begin{theorem}
    When the initial density $p_0(x)$ is known, the density field can be expressed as:
\begin{equation}
    p(\phi_t(x), t) = \frac{p_0(x)}{\left|\det J_{\phi_t}(x)\right|}
\end{equation}
\end{theorem}

It should be noted that $\phi_t(x)$ is actually the same as $X_t$ in SDE, then similarly, the density is:
\begin{equation}
    \phi_t(x) \sim p_t(x)
\end{equation}
So, the flow map is an ODE, which is a special case of SDE without diffusion term. Then we have:
\begin{theorem}[Continuity Equation]
    The probability density function $p(x, t)$ of $X_t$ satisfies:
    \begin{equation}
        \frac{\partial p(x, t)}{\partial t} = -\nabla\cdot\left(V(x, t)p(x, t)\right)
    \end{equation}
    which is called \textbf{Continuity Equation}.
\end{theorem}
\begin{remark}
    The continuity equation can also be derived from the Conservation of Mass. 
\end{remark}

\begin{theorem}
    When the incompressible condition is satisfied, that is $\nabla\cdot V=0$, then the flow $\phi_t(x)$ is \textbf{measure preserving}, that is:
    \begin{equation}
        \left|\det J_{\phi_t}(x)\right|=1, \text{i.e.}p(\phi_t(x), t) = p_0(x)
    \end{equation}
\end{theorem}

\begin{definition}[Flux]
    We find that $V(x, t)p(x, t)$ is actually the flux $\mathcal{F}(x, t)$ of the particle.
\end{definition}
Then the continuity equation can be rewritten as:
\begin{equation}
    \frac{\partial p(x, t)}{\partial t} = -\nabla\cdot\left(\mathcal{F}(x, t)\right)
\end{equation}
Then we find that if the flux s.t. $F = -\frac{1}{2}\nabla\cdot\left(GQG^Tp(x, t)\right)$, then $p(x, t)$ describes the diffusion process. This is the famous Fick's Law.
\begin{theorem}[Fick's Law]
    Fick's Law describes the relationship between the flux $\mathcal{F}(x, t)$ of the particle and the concentration/density $p(x, t)$.:
    \begin{equation}
        \mathcal{F}(x, t) = -\frac{1}{2}\nabla\cdot\left(GQG^Tp(x, t)\right)
    \end{equation}
    Specifically, when $G(X_t, t)=G(t)$ and $Q=I$, we have:
    \begin{equation}
        \mathcal{F}(x, t) = -\frac{GG^T}{2}\nabla p(x, t)
    \end{equation}
    Then 
    \begin{equation}
        \frac{\partial p(x, t)}{\partial t} = \nabla \cdot\left(\frac{GG^T}{2}\nabla p(x, t)\right)
    \end{equation}
\end{theorem}
\subsection{Solution}
Note $-\frac{GG^T}{2}$ is actually the diffusion coefficient $\mathcal{D}$. Then we have the diffusion equation:
\begin{equation}
    \frac{\partial p(x, t)}{\partial t} = \nabla \cdot\left(\mathcal{D}\nabla p(x, t)\right)
\end{equation}
with initial condition $p(x, 0) = p_0(x)$. We can use the Fourier Transform to solve this equation. 
\begin{theorem}
    The solution to the diffusion equation is:
    \begin{equation}
        \begin{aligned}
            p(x, t) &= \mathscr{F}^{-1}\left[\tilde{p}_0(\lambda)\exp\left(-\lambda^T\mathcal{D}\lambda t\right)\right]=\left(p_0\star \mathcal{G}_{2t\mathcal{D}}\right)(x)\\
            & = \frac{1}{\sqrt{(4\pi t)^d\det(\mathcal{D})}}\int_{\mathcal{R}^d}\left(p_0(\xi)\exp\left(-\frac{1}{4t}\left(x-\xi\right)^T\mathcal{D}^{-1}\left(x-\xi\right)\right)\right)d\xi
        \end{aligned}
    \end{equation}
    where $\tilde{p}_0(\lambda) = \mathscr{F}(p_0(x))$ is the Fourier Transform of $p_0(x)$. $\mathcal{G}_{2t\mathcal{D}}$ is the Gaussian Kernel with variance $2t\mathcal{D}$.
\end{theorem}
\begin{proof}
    First, assume the Fourier Transform of $p(x, t)$ is $\tilde{p}(x, t)$:
    \begin{equation}\left\{
        \begin{aligned}
            \tilde{p}(x, t) &= \mathscr{F}\left[p(x, t)\right]=\int_{\mathcal{R}^d} p(x, t)e^{-i\lambda\cdot x}dx\\
            p(x, t) &= \mathscr{F}^{-1}\left[\tilde{p}(x, t)\right]=\frac{1}{(2\pi)^d}\int_{\mathcal{R}^d} \tilde{p}(x, t)e^{i\lambda\cdot x}dx
        \end{aligned}\right.
    \end{equation} 
    Then, we have:
    \begin{equation}\left\{
        \begin{aligned}
            \mathscr{F}\left[\nabla\cdot \mathbf{v}\right] &= i\lambda\cdot \mathscr{F}\left[\mathbf{v}\right]\\
            \mathscr{F}\left[\mathcal{D}\nabla p\right] &= i\mathcal{D}\lambda\mathscr{F}\left[p\right]
        \end{aligned}\right.
    \end{equation}
    Then, 
    \begin{equation}
        \begin{aligned}
            &\mathscr{F}\left[\frac{\partial p}{\partial t}\right] = \frac{d}{dt}\mathscr{F}\left[p\right] = \mathscr{F}\left[\nabla\cdot\left(\mathcal{D}\nabla p\right)\right] \\
            =& i\lambda\cdot \mathscr{F}\left[\mathcal{D}\nabla p\right]=-\lambda^T\mathcal{D}\lambda \mathscr{F}\left[p\right]
        \end{aligned}
    \end{equation}
    where $\lambda = \left(\lambda_1, \lambda_2, \cdots, \lambda_d\right)^T$. 

    Therefore, $\mathscr{F}\left[p\right] = \tilde{p_0}\exp\left(-\lambda^T\mathcal{D}\lambda t\right)$. 
    Since $\mathscr{F}\left[N(x|0, 2t\mathcal{D})\right]=\exp\left(-\lambda^T\mathcal{D}\lambda t\right)$, which gives the theorem.
\end{proof}

\begin{remark}
    Specially, 1. When the initial density $p_0(x)$ is $\delta(x - x_0)$, the solution is:
\begin{equation}
    p(x, t) = \frac{1}{\sqrt{(4\pi t)^d\det{\mathcal{D}}}}\exp\left(-\frac{(x-x_0)^T\mathcal{D}^{-1}(x-x_0)}{4t}\right)\sim N(x_0, 2t\mathcal{D})
\end{equation}
2. When the initial density $p_0(x)$ is a Gaussian distribution $N(\mu, \Sigma)$, the solution is:
\begin{equation}
    p(x, t) = \frac{1}{\sqrt{(2\pi)^d\det(\Sigma + 2t\mathcal{D})}}\exp\left(-\frac{1}{2}\left(x-\mu\right)^T\left(\Sigma + 2t\mathcal{D}\right)^{-1}\left(x-\mu\right)\right)\sim N(\mu, \Sigma + 2t\mathcal{D})
\end{equation}
(The Fourier transform of $\left(\mu, \Sigma \right)$ is $\exp \left(-i\lambda^T\mu + \frac{1}{2}\lambda^T\Sigma\lambda\right)$.)
\end{remark}

Till here, we can see the insight of diffusion. It is actually a process of smoothing the initial density by the Gaussian Kernel.


\section{Reversible Diffusions}
\subsection{Definition}
\begin{definition}[Time-reversible]
    A stationary stochastic process $X_t$ is called time-reversible 
    if for every $T\in (0, +\infty)$, the process $X_{T-t}$ has the same distribution as $X_t$.
\end{definition}

\begin{theorem}
    A stationary Markov process $X_t$ in $\mathbb{R}^d$ with generator $\mathcal{L}$ and invariant measure $\mu$ is time-reversible 
    if and only if $\mathcal{L}$ is self-adjoint in $L^2(\mathbb{R}^d; \mu)$.
\end{theorem}

Since for general SDE (\ref{sde}), the generator operator $\mathcal{L}$ and its self adjoint operator $\mathcal{L}^*$ are given by:
\begin{equation}\left\{
\begin{aligned}
    \mathcal{L}(\cdot)&=\sum_{i} \frac{\partial(\cdot)}{\partial x_{i}} f_{i}(x, t)+\frac{1}{2} \sum_{i, j}\left(\frac{\partial^{2}(\cdot)}{\partial x_{i} \partial x_{j}}\right)\left[GQ G^{\top}\right]_{i j}\\
    &= f\cdot \nabla(\cdot) + \frac{1}{2} \left(M:\nabla\cdot \nabla\right)(\cdot)\\
    \mathcal{L}^*(\cdot)&=-\sum_{i} \frac{\partial}{\partial x_{i}}\left(f_{i}(x, t) (\cdot)\right)+\frac{1}{2} \sum_{i=1} \frac{\partial^{2}}{\partial x_{i} \partial x_{j}}\left(\left[G Q G^{\top}\right]_{i j}(\cdot)\right)\\
    & = \nabla\cdot (-f(\cdot)+\frac{1}{2}\nabla\cdot(M(\cdot)))
\end{aligned}\right.
\end{equation}
We assume that the diffusion process has a unique invariant distribution which is the solution of the stationary Fokker-Planck equation:
\begin{equation}
    \mathcal{L}^*\rho_s=0
\end{equation}
Notice that we can write the invariant distribution $\rho_s$ in the form:
\begin{equation}
    \rho_s = e^{-\Phi}
\end{equation}
where $\Phi$ is a potential function.
\begin{theorem}
    For stationary process $X_t$ with invariant distribution $\rho_s$.
    To guarantee the operator $\mathcal{L}$ is symmetric if and only if $J(\rho_s)=0$. This is the detailed balance condition.
    So, expand the stationary probability flux:
    \begin{equation}
        f = \frac{1}{2}\rho_s^{-1}\nabla\cdot\left(M\rho_s\right)
    \end{equation}
\end{theorem}

Consider now an arbitrary ergodic diffusion process $X_t$, the solution of (\ref{sde}) with invariant distribution $\rho_s$.
We can decompose this process into a reversible and a nonreversible part in the sense that the generator  can be decomposed into a symmetric and antisymmetric part.

\begin{theorem}
    The generator of an arbitrary diffusion process in $\mathbb{R}^d$ can we written as:
    \begin{equation}
        \mathcal{L} = \rho_s^{-1}J_s\cdot \nabla + \frac{1}{2}\rho_s^{-1}\nabla\cdot\left(M\rho_s\nabla\right):=\mathcal{S}+\mathcal{A}
    \end{equation}
    where $\mathcal{S}$ is the symmetric part of $\mathcal{L}$ and $\mathcal{A}$ is the antisymmetric part of $\mathcal{L}$.
\end{theorem}

\begin{example}
    When $M = 2I$, then 
    \begin{equation}
        f = \rho^{-1}\nabla\rho = \nabla \log \rho
    \end{equation}
    which is the form of Langevin equation.
\end{example}
\subsection{Schrödinger Operator}
...

