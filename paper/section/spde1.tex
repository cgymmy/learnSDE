\section{Stationary SPDEs}
\subsection{Definition}
\begin{definition}[Stationary SPDE]
    Assume given $a, f\in L^2(\Omega, L^2(D))$ are random fields, try to seek $u:\bar{D}\times \Omega \to \mathbb{R}$ in weak sense s.t. $\mathbb{P}$-a.s.:
    \begin{equation}\left\{
        \begin{aligned}
            -\nabla \cdot (a(x, w)\nabla u(x, w)) = f(x, w),\qquad x\in D\\
            u(x, w) = g(x),\qquad x\in \partial D
        \end{aligned}\right.\label{spde1model}
    \end{equation}
    To ensure the existence of solution, we need to impose some conditions on $g$.
\end{definition}

\begin{definition}[Weak solution on $D\times \Omega$]
    A weak solution to Eq(\ref{spde1model}) with $g=0$ is a function $u\in V=L^2(\Omega, H_0^1(D))$ s.t. for any $v\in V$,
    \begin{equation}
        a(u, v) = l(v)\label{spde1weak}
    \end{equation}
    where\begin{equation}\left\{
        \begin{aligned}
            a(u, v) &= E\left[\int_D a(x, \cdot)\nabla u(x, \cdot)\cdot \nabla v(x, \cdot)dx\right]\\
            l(v) &= E\left[\int_D f(x, \cdot)v(x, \cdot)dx\right]
        \end{aligned}\right.
    \end{equation}
    If $g\neq 0$, a weak solution to Eq(\ref{spde1model}) is a function $u\in W=L^2(\Omega, H_g^1(D))$ s.t. for any $v\in V$,
    \begin{equation}
        a(u, v) = l(v)\label{spde1weaknh}
    \end{equation}
    where $a(\cdot, \cdot):W\times V\to \mathbb{R}$ and $l:V\to \mathbb{R}$:
    \begin{equation}\left\{
        \begin{aligned}
            a(u, v) &= E\left[\int_D a(x, \cdot)\nabla u(x, \cdot)\cdot \nabla v(x, \cdot)dx\right]\\
            l(v) &= E\left[\int_D f(x, \cdot)v(x, \cdot)dx\right]
        \end{aligned}\right.
    \end{equation}
\end{definition}

\begin{theorem}[Existence and uniqueness of weak solution]
    Note for all $x\in D$
    \begin{equation}\label{assumption1}
        0<a_{\min}\leq a(x, \cdot)\leq a_{\max}<\infty
    \end{equation} as a basic assumption.

    If $f\in L^2(\Omega, L^2(D)),g=0$, and Assumption (\ref{assumption1}) holds, then SPDE \ref{spde1weak} has a unique weak solution $u\in V$.

    If Assumption (\ref{assumption1}) holds, $f\in L^2(\Omega, L^2(D))$, and $g\in H^{\frac{1}{2}}(\partial D)$, 
    then SPDE \ref{spde1weaknh} has a unique weak solution $u\in W$.
\end{theorem}

Assume we have the approximate random fields $\tilde{a}, \tilde{f}: D\times \Omega \to \mathbb{R}$ s.t. (\ref{assumption1}) holds.

Then as mentioned before, we can expand $a, f$ in terms of (truncated) Karhunen-Loeve expansion as:
\begin{equation}\left\{
    \begin{aligned}
        &a(x, w) = \mu_a(x) + \sum_{i=1}^{N_a}\sqrt{v_i^a}\phi_i^a(x)\xi_i^a(w)\\
        &f(x, w) = \mu_f(x) + \sum_{i=1}^{N_f}\sqrt{v_i^f}\phi_i^f(x)\xi_i^f(w)
    \end{aligned}\right.\label{expansion}
\end{equation}
where $(v_i^a, \phi_i^a), (v_i^f, \phi_i^f)$ are the eigenpairs of the covariance operators of $a, f$ respectively, 
and $\xi_i^a, \xi_i^f$ are i.i.d. random variables.

The next question is how to compute:
\begin{equation}
    \begin{aligned}
        a(u, v) &= E\left[\int_D a(x, \cdot)\nabla u(x, \cdot)\cdot \nabla v(x, \cdot)dx\right]\\
        &=\int_\Omega \int_D a(x, w)\nabla u(x, w)\cdot \nabla v(x, w)dxdP(w)\\
    \end{aligned}
\end{equation}

Since the truncated KL expansion of $a(x, w)$ depends on a finite number $N_a$ of random variables $\xi_i^a:\Omega\to \Gamma_i$(same as $f(x,w)$), 
we consider weak form of Eq(\ref{spde1model}) on $D\times \Gamma$, where $\Gamma = \prod_{i=1}^{N_a}\Gamma_i$.

\begin{definition}[finite-dimensional noise]
    A function $v\in L^2(\Omega, L^2(D))$ of the form $v(x, \xi(w))$ for $\forall x\in D,w\in \Omega$, where $\xi = [\xi_1, \cdots, \xi_{N}]^T:\Omega\to \Gamma$, is called a finite-dimensional noise.
\end{definition}

\begin{definition}[Weak solution on $D\times \Gamma$]
    Let $\tilde{a}(x)$ and $\tilde{f}(x)$ be finite-dimensional noises defined in Eq(\ref{expansion}), then the solution to Eq (\ref{spde1model}) is also finite-dimensional noise. 
    Define
    \begin{equation}
        W:=L^2_p(\Gamma, H^1_g(D)) = \left\{v:D\times \Gamma\to \mathbb{R}: \int_\Gamma \|v(\xi, \cdot)\|_{H^1_g(D)}^2d\xi<\infty\right\}
    \end{equation}
    A weak solution to Eq(\ref{spde1model}) on $D\times \Gamma$ is a function $u\in W=L^2_p(\Gamma, H^1_g(D))$ s.t. for any $v\in V=L^2_p(\Gamma, H^1_0(D))$,
    \begin{equation}
        a(u, v) = l(v)\label{weakfd}
    \end{equation}
    where
    \begin{equation}\left\{
        \begin{aligned}
            a(u, v) &= \int_\Gamma p(\xi)\int_D \tilde{a}(x, \xi)\nabla u(x, \xi)\cdot \nabla v(x, \xi)dxd\xi\\
            l(v) &= \int_\Gamma p(\xi)\int_D \tilde{f}(x, \xi)v(x, \xi)dxd\xi
        \end{aligned}\right.
    \end{equation}
\end{definition}
\subsection{Stochastic Galerkin Method}
Therefore, we have the stochastic Galerkin solution: seek $u_{hk}\in W^{hk}\subset L^2(\Gamma, H^1_g(D))$ s.t. for any $v_{hk}\in V^{hk}\subset L^2(\Gamma, H^1_0(D))$.

By define the inner product:
\begin{equation}
    \langle v, w\rangle_{p} = \int_\Gamma v(\xi)w(\xi)P(\xi)d\xi
\end{equation}
We can construct a sequence of polynomials $P_i(\xi)$ on $\Gamma$. Hence:
\begin{equation}
    L^2_p(\Gamma):=\{v:\Gamma\to \mathbb{R}: \|v\|^2_{L^2_p(\Gamma)} = \langle v, v\rangle_p<\infty\}
\end{equation}

\begin{definition}
    Note $S^k$ be the set of polynomials of degree $k$ or less on $\Gamma$:
    \begin{equation}
        \begin{aligned}
            S^k &= \operatorname{span}\{\prod_{i=1}^{M}P_i^{\alpha_i}(\xi_i): \alpha_i\in \mathbb{N}_0, \sum_{i=1}^{M}\alpha_i\leq k\}\\
            &= \operatorname{span}\{\psi_1, \psi_2, \cdots, \psi_Q\}
        \end{aligned}
    \end{equation}
    where $P_i(\xi_i)$ is some polynomial. And $Q=\operatorname{dim}S^k = \binom{M+k}{k}$.
\end{definition}
We need $S^k\subset L^2_p(\Gamma)$ where $\Gamma \subset \mathbb{R}^M$. If $\{\xi_i\}$ are independent, then the joint density $p$ is:
\begin{equation}
    p(\xi) = \prod_{i=1}^{M}p_i(\xi_i)
\end{equation}
Recall $V^h=\operatorname{span}\{\phi_i\}_{i=1}^{J}\subset H^1_0(D)$ is the finite element space, we have tensor product space:
\begin{equation}
    V^{hk}:= V^h\otimes S^k = \operatorname{span}\{\phi_i\psi_j\}_{i=1, j=1}^{J,Q}
\end{equation}

Then 
\begin{equation}
    W^{hk}:=V^{hk}\oplus \operatorname{span}\{\phi_{J+1},\cdots, \phi_{J+J_{b}}\}
\end{equation}
where $J_b$ is finite element functions associated with Dirichlet boundary vertices.

\begin{theorem}[Stochastic basis functions]
    If $\{\xi_i\}$ are independent, suppose that $\{P_i^{\alpha_i}(\xi_i)\}_{\alpha_i=1}^{M}$ are orthonormal with $\langle\cdot,\cdot\rangle_{p_i}$ on $\Gamma_i$.
    Then the complete orthonormal polynomials $\{\psi_j\}_{j=1}^{Q}$ are orthonormal with $\langle\cdot,\cdot\rangle_p$ on $\Gamma$.
\end{theorem}

Then $u_{hk}$ can be written as:
\begin{equation}
    u_{hk}(x, \xi) = \sum_{i=1}^{J}\sum_{j=1}^{Q}u_{ij}\phi_i(x)\psi_j(\xi)+w_g
\end{equation}

\begin{theorem}[Mean and covariance]
    The Galerkin solution can be rewritten as:
    \begin{equation}
        \begin{aligned}
            u_{hk}(x, \xi) &=\sum_{j=1}^{Q}\left(\sum_{i=1}^{J}u_{ij}\phi_i(x)\right)\psi_j(\xi)+w_g\\
            &=\sum_{j=1}^{Q}u_j\psi_j(\xi)+w_g\\
            &= (u_1(x)+w_g(x))\psi_1(\xi) + \sum_{j=2}^{Q}u_j(x)\psi_j(\xi)\\
        \end{aligned}
    \end{equation}
    Then the mean and covariance is 
    \begin{equation}\left\{
        \begin{aligned}
            E[u_{hk}] &= u_1+w_g\\
            Var(u_{hk}) &= \sum_{j=2}^{Q}u_j^2
        \end{aligned}\right.
    \end{equation}
\end{theorem}
\subsection{Algorithm}
Expand $u_{hk}$ in terms of basis functions $v = \phi_r \psi_s$ for $r=1,2,\cdots, J, s=1,2,\cdots, Q$, we have the linear system:
\begin{equation}
    A = \begin{pmatrix}
        A_{11} & A_{12} & \cdots & A_{1Q}\\
        A_{21} & A_{22} & \cdots & A_{2Q}\\
        \vdots & \vdots & \ddots & \vdots\\
        A_{Q1} & A_{Q2} & \cdots & A_{QQ}
    \end{pmatrix},
    \mathbf{u} = \begin{pmatrix}
        \mathbf{u}_1\\
        \mathbf{u}_2\\
        \vdots\\
        \mathbf{u}_Q
    \end{pmatrix},
    \mathbf{b} = \begin{pmatrix}
        \mathbf{b}_1\\
        \mathbf{b}_2\\
        \vdots\\
        \mathbf{b}_Q
    \end{pmatrix}
\end{equation}
where
\begin{equation}
    \mathbf{u}_j = [u_{1j}, u_{2j}, \cdots, u_{Jj}]^T, j=1,2,\cdots, Q
\end{equation}
and each submatrix $A_{ij}$ is a $J\times J$ matrix, $i,j=1,2,\cdots, Q$:
\begin{equation}
    A_{ij} = K_0\langle \psi_i, \psi_j\rangle_p + \sum_{l=1}^{P}K_l\langle \psi_i, \psi_j\xi_l\rangle_p
\end{equation}
where
\begin{equation}\left\{
    \begin{aligned}
        \left[K_0\right]_{rs} &= \int_D \mu_a(x)\nabla\phi_r\cdot \nabla\phi_sdx\\
        \left[K_l\right]_{rs} &= \int_D (\sqrt{v_l^a}\phi_l^a)\nabla\phi_r\cdot \nabla\phi_sdx
    \end{aligned}\right., \qquad r,s=1,2,\cdots, J
\end{equation}

And $\mathbf{b}_s$ is a $J\times 1$ vector:
\begin{equation}
    \mathbf{b}_s = \langle \psi_1, \psi_s\rangle_pF_0 + \sum_{l=1}^{P}F_l\langle \xi_l, \psi_s\rangle_p - \langle W, \psi_s\rangle_p
\end{equation}
where
\begin{equation}\left\{
    \begin{aligned}
        \left[F_0\right]_{i} &= \int_D \mu_f(x)\phi_i(x)dx\\
        \left[F_l\right]_{i} &= \int_D (\sqrt{v_l^f}\phi_l^f)\phi_i(x)dx\\
        W &= K^T_{0B}\mathbf{w}_B+\sum_{l=1}^{P}K^T_{lB}\xi_l\mathbf{w}_B
    \end{aligned}\right.
\end{equation}


























