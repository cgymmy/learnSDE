\section{What is Diffusion After All?}
\subsection{From SDEs}
At the beginning, the diffusion phenomenon is observed through the motion of particles(Brownian motion). 
Normally, the SDE can be written as:
\begin{equation}
    dX_t = f(X_t, t)dt + G(X_t, t)dW_t
\end{equation}
Here, we skip the drift term $f(X_t, t)$ and only consider the diffusion term $G(X_t, t)dW_t$, i.e.
\begin{equation}
    dX_t = G(X_t, t)dW_t
\end{equation}
Then by FPK equation, we can derive 
\begin{theorem}
    The probability density function $p(x, t)$ satisfies:
\begin{equation}
    \frac{\partial p(x, t)}{\partial t} = \frac{1}{2} \sum_{i, j} \frac{\partial^{2}}{\partial x_{i} \partial x_{j}}\left[\left(G Q G^{\top}\right)_{i j} p(x, t)\right]=\frac{1}{2}\nabla \cdot \left(\nabla\cdot (GQG^Tp(x, t))\right)
\end{equation}
Specially, when $G(X_t, t)=G(t)$ and $Q=I$, we have:
\begin{equation}
    \frac{\partial p}{\partial t} = \nabla \cdot \left(\frac{GG^T}{2}\nabla p\right)
\end{equation}
\end{theorem}
So, when $X_0\sim p_0$, we can then compute the diffusion density $p(x, t)$ by solving the FPK equation.
\subsection{From Flow Map}
Since we have the definition of \textbf{Flow Map} $\phi_s^t(\mathbf{x})$, which is controlled by vector field $V(\phi_s^t(\mathbf{x}), t)$, 
then just think the $\phi_0^t(\mathbf{x})$ as the trajectory of the particle beginning at $x$ over time, noted as $\phi_t(x)$.
Then the vector field is actually the velocity field of the particle, so we have:
\begin{equation}\left\{
    \begin{aligned}
        \frac{\partial \phi_t(\mathbf{x})}{\partial t} &= V(\phi_t(\mathbf{x}), t)\\
        \phi_0(\mathbf{x}) &= \mathbf{x}
    \end{aligned}\right.
\end{equation}
The motion of particles described by $\phi_t$ determines how the density $p_t(x)$ evolves over time. 
\begin{theorem}
    When the initial density $p_0(x)$ is known, the density field can be expressed as:
\begin{equation}
    p(\phi_t(x), t) = \frac{p_0(x)}{\left|\det J_{\phi_t}(x)\right|}
\end{equation}
\end{theorem}

It should be noted that $\phi_t(x)$ is actually the same as $X_t$ in SDE, then similarly, the density is:
\begin{equation}
    \phi_t(x) \sim p_t(x)
\end{equation}
So, the flow map is an ODE, which is a special case of SDE without diffusion term. Then we have:
\begin{theorem}[Continuity Equation]
    The probability density function $p(x, t)$ of $X_t$ satisfies:
    \begin{equation}
        \frac{\partial p(x, t)}{\partial t} = -\nabla\cdot\left(V(x, t)p(x, t)\right)
    \end{equation}
    which is called \textbf{Continuity Equation}.
\end{theorem}
\begin{remark}
    The continuity equation can also be derived from the Conservation of Mass. 
\end{remark}

\begin{theorem}
    When the incompressible condition is satisfied, that is $\nabla\cdot V=0$, then the flow $\phi_t(x)$ is \textbf{measure preserving}, that is:
    \begin{equation}
        \left|\det J_{\phi_t}(x)\right|=1, \text{i.e.}p(\phi_t(x), t) = p_0(x)
    \end{equation}
\end{theorem}

\begin{definition}[Flux]
    We find that $V(x, t)p(x, t)$ is actually the flux $\mathcal{F}(x, t)$ of the particle.
\end{definition}
Then the continuity equation can be rewritten as:
\begin{equation}
    \frac{\partial p(x, t)}{\partial t} = -\nabla\cdot\left(\mathcal{F}(x, t)\right)
\end{equation}
Then we find that if the flux s.t. $F = -\frac{1}{2}\nabla\cdot\left(GQG^Tp(x, t)\right)$, then $p(x, t)$ describes the diffusion process. This is the famous Fick's Law.
\begin{theorem}[Fick's Law]
    Fick's Law describes the relationship between the flux $\mathcal{F}(x, t)$ of the particle and the concentration/density $p(x, t)$.:
    \begin{equation}
        \mathcal{F}(x, t) = -\frac{1}{2}\nabla\cdot\left(GQG^Tp(x, t)\right)
    \end{equation}
    Specifically, when $G(X_t, t)=G(t)$ and $Q=I$, we have:
    \begin{equation}
        \mathcal{F}(x, t) = -\frac{GG^T}{2}\nabla p(x, t)
    \end{equation}
    Then 
    \begin{equation}
        \frac{\partial p(x, t)}{\partial t} = \nabla \cdot\left(\frac{GG^T}{2}\nabla p(x, t)\right)
    \end{equation}
\end{theorem}
\subsection{Solution}
Note $-\frac{GG^T}{2}$ is actually the diffusion coefficient $\mathcal{D}$. Then we have the diffusion equation:
\begin{equation}
    \frac{\partial p(x, t)}{\partial t} = \nabla \cdot\left(\mathcal{D}\nabla p(x, t)\right)
\end{equation}
with initial condition $p(x, 0) = p_0(x)$. We can use the Fourier Transform to solve this equation. 
\begin{theorem}
    The solution to the diffusion equation is:
    \begin{equation}
        \begin{aligned}
            p(x, t) &= \mathscr{F}^{-1}\left[\tilde{p}_0(\lambda)\exp\left(-\lambda^T\mathcal{D}\lambda t\right)\right]=\left(p_0\star \mathcal{G}_{2t\mathcal{D}}\right)(x)\\
            & = \frac{1}{\sqrt{(4\pi t)^d\det(\mathcal{D})}}\int_{\mathcal{R}^d}\left(p_0(\xi)\exp\left(-\frac{1}{4t}\left(x-\xi\right)^T\mathcal{D}^{-1}\left(x-\xi\right)\right)\right)d\xi
        \end{aligned}
    \end{equation}
    where $\tilde{p}_0(\lambda) = \mathscr{F}(p_0(x))$ is the Fourier Transform of $p_0(x)$. $\mathcal{G}_{2t\mathcal{D}}$ is the Gaussian Kernel with variance $2t\mathcal{D}$.
\end{theorem}
\begin{proof}
    First, assume the Fourier Transform of $p(x, t)$ is $\tilde{p}(x, t)$:
    \begin{equation}\left\{
        \begin{aligned}
            \tilde{p}(x, t) &= \mathscr{F}\left[p(x, t)\right]=\int_{\mathcal{R}^d} p(x, t)e^{-i\lambda\cdot x}dx\\
            p(x, t) &= \mathscr{F}^{-1}\left[\tilde{p}(x, t)\right]=\frac{1}{(2\pi)^d}\int_{\mathcal{R}^d} \tilde{p}(x, t)e^{i\lambda\cdot x}dx
        \end{aligned}\right.
    \end{equation} 
    Then, we have:
    \begin{equation}\left\{
        \begin{aligned}
            \mathscr{F}\left[\nabla\cdot \mathbf{v}\right] &= i\lambda\cdot \mathscr{F}\left[\mathbf{v}\right]\\
            \mathscr{F}\left[\mathcal{D}\nabla p\right] &= i\mathcal{D}\lambda\mathscr{F}\left[p\right]
        \end{aligned}\right.
    \end{equation}
    Then, 
    \begin{equation}
        \begin{aligned}
            &\mathscr{F}\left[\frac{\partial p}{\partial t}\right] = \frac{d}{dt}\mathscr{F}\left[p\right] = \mathscr{F}\left[\nabla\cdot\left(\mathcal{D}\nabla p\right)\right] \\
            =& i\lambda\cdot \mathscr{F}\left[\mathcal{D}\nabla p\right]=-\lambda^T\mathcal{D}\lambda \mathscr{F}\left[p\right]
        \end{aligned}
    \end{equation}
    where $\lambda = \left(\lambda_1, \lambda_2, \cdots, \lambda_d\right)^T$. 

    Therefore, $\mathscr{F}\left[p\right] = \tilde{p_0}\exp\left(-\lambda^T\mathcal{D}\lambda t\right)$. 
    Since $\mathscr{F}\left[N(x|0, 2t\mathcal{D})\right]=\exp\left(-\lambda^T\mathcal{D}\lambda t\right)$, which gives the theorem.
\end{proof}

\begin{remark}
    Specially, 1. When the initial density $p_0(x)$ is $\delta(x - x_0)$, the solution is:
\begin{equation}
    p(x, t) = \frac{1}{\sqrt{(4\pi t)^d\det{\mathcal{D}}}}\exp\left(-\frac{(x-x_0)^T\mathcal{D}^{-1}(x-x_0)}{4t}\right)\sim N(x_0, 2t\mathcal{D})
\end{equation}
2. When the initial density $p_0(x)$ is a Gaussian distribution $N(\mu, \Sigma)$, the solution is:
\begin{equation}
    p(x, t) = \frac{1}{\sqrt{(2\pi)^d\det(\Sigma + 2t\mathcal{D})}}\exp\left(-\frac{1}{2}\left(x-\mu\right)^T\left(\Sigma + 2t\mathcal{D}\right)^{-1}\left(x-\mu\right)\right)\sim N(\mu, \Sigma + 2t\mathcal{D})
\end{equation}
(The Fourier transform of $\left(\mu, \Sigma \right)$ is $\exp \left(-i\lambda^T\mu + \frac{1}{2}\lambda^T\Sigma\lambda\right)$.)
\end{remark}

Till here, we can see the insight of diffusion. It is actually a process of smoothing the initial density by the Gaussian Kernel.
